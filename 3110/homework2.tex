\documentclass[11pt]{amsart}

\usepackage{amssymb,amsmath}
\usepackage{bm}
\usepackage{mathtools}
\usepackage{framed}
\usepackage{esvect}
\usepackage{amsthm}
\usepackage{centernot}
\usepackage{ifxetex,ifluatex}

%%%%%%%%%%%%%%% FILL THIS IN FOR EACH ASSIGNMENT
\newcommand{\name}{Kyrylo Chernyshov}
\newcommand{\sectionnum}{203}
\newcommand{\norm}[1]{\left\lVert#1\right\rVert}
\newcommand*\diff{\mathop{}\!\mathrm{d}}
\newcommand*\Diff[1]{\mathop{}\!\mathrm{d^#1}}
\newcommand{\ex}{\subsubsection{Example}}
\newcommand*\mean[1]{\bar{#1}}

\DeclareMathOperator{\proj}{proj}
\DeclareMathOperator{\im}{im}
\DeclareMathOperator{\row}{row}
\DeclareMathOperator{\col}{col}
\DeclareMathOperator{\rank}{rank}
\DeclareMathOperator{\nullity}{nullity}
\DeclareMathOperator{\detm}{det}
\DeclarePairedDelimiter\ceil{\lceil}{\rceil}
\DeclarePairedDelimiter\floor{\lfloor}{\rfloor}
%%%%%%%%%%%%%%%%%%%%%%%%%%%%%%%%%%%%%%%%%%%%%%%%

\usepackage[margin=1in, letterpaper]{geometry}
\newcommand{\problem}[1]{\bigskip\noindent\textbf{Problem #1}}
\newcommand{\ppart}[1]{\bigskip\textbf{(#1)}}
\newcommand{\lemma}{\bigskip\textbf{Lemma}}
\newcommand{\E}{\mathrm{E}}
\newcommand{\Var}{\mathrm{Var}}
\newcommand{\Cov}{\mathrm{Cov}}

\begin{document}
\title{Math 3110 homework \#2}
\author{\name}
\maketitle

\problem{2.1.3}
This can easily be disproven by means of a counterexample. Suppose $a_n = n$, and $b_n = n - 3$, both clearly increasing sequences. Then, $a_nb_n = n(n - 4)$, that is, $a_1b_1 = -3$, $a_2b_2 = -4$ and $a_3b_3 = -3$. Thus, $a_nb_n$ is neither increasing nor decreasing.

A better statement is that if $\{a_n\}$ and $\{b_n\}$ are increasing \textbf{and} $a_n, b_n > 0 \;\forall\; n$, then $\{a_nb_n\}$ is increasing.

\begin{proof}
  Suppose $\{a_n\}$ and $\{b_n\}$ are increasing, that is, $a_{n + 1} \geq a_n$ and $b_{n + 1} \geq b_n$ for all $n$. Consider the ratio between two consecutive terms of $\{a_nb_n\}$:

  \begin{align*}
    \frac{a_{n + 1}b_{n + 1}}{a_nb_n} = \frac{a_{n + 1}}{a_n} \cdot \frac{b_{n + 1}}{b_n}
  \end{align*}

  Since $\{a_n\}$ and $\{b_n\}$ are increasing with all positive terms, we know that $\frac{a_{n + 1}}{a_n} \geq 0$ and $\frac{b_{n + 1}}{b_n} \geq 0$. Therefore, $\frac{a_{n + 1}b_{n + 1}}{a_nb_n} \geq 0$.
\end{proof}

\problem{2.4.2}
The contrapositive of this statement is as follows: if $|a_i| \leq 1 \;\forall\; i$, then

\begin{align*}
  \left|\sum_{i = 1}^n a_i \sin (ib)\right| \leq n
\end{align*}

\begin{proof}
  Assume that $|a_i| \leq 1 \;\forall\; i$. A fundamental fact is that $|\sin x| \leq 1 \;\forall\; x \in \mathbb{R}$. Therefore, $|a_i \sin (ib)| \leq 1 \;\forall\; i$.

  By the triangle inequality, $\left|\sum_{i = 1}^n a_i \sin (ib)\right| \leq \sum_{i = 1}^n |a_i \sin (ib)|$. Since $|a_i \sin (ib)| \leq 1$, $\sum_{i = 1}^n |a_i \sin (ib)| \leq \sum_{i = 1}^n 1 = n$.
\end{proof}

\problem{2.6.1}
To put this statement formally, there exists $N \in \mathbb{N}$ such that for all $n \geq N$, $\frac{a^{n + 1}}{(n + 1)!} > \frac{a^n}{n!}$.

\begin{proof}
  Let $N$ be the smallest integer such that $N \geq a$. Then, consider the ratio between two consecutive terms of the sequence:

  \begin{align*}
    \frac{a_{n + 1}}{a_n} = \frac{a^{n + 1}}{(n + 1)!} \cdot \frac{n!}{a^n} &= \frac{a^{n + 1}}{a^n} \cdot \frac{n!}{(n + 1)!} \\
    &= a \cdot \frac{1}{n + 1} = \frac{a}{n + 1}
  \end{align*}

  Since $n \geq N \geq a$, $n + 1 > a$, and since $a > 0$, $0 < \frac{a}{n + 1} < 1$. Since the ratio between two consecutive terms is positive but less than 1, the sequence must be monotone strictly decreasing.
\end{proof}

\problem{2-1}

\ppart{a}
\begin{proof}
  Consider the difference between two consecutive terms:

  \begin{align*}
    b_{n + 1} - b_n &= \frac{\sum_{i = 1}^{n + 1} a_i}{n + 1} - \frac{\sum_{i = 1}^{n} a_i}{n} \\
    &= \frac{n\sum_{i = 1}^{n + 1} a_i - (n + 1)\sum_{i = 1}^{n} a_i}{n(n + 1)} \\
    &= \frac{na_{n + 1} - \sum_{i = 1}^{n} a_i}{n(n + 1)} \\
    &= \frac{\sum_{i = 1}^{n}a_{n + 1} - \sum_{i = 1}^{n} a_i}{n(n + 1)} = \frac{\sum_{i = 1}^{n}(a_{n + 1} - a_i)}{n(n + 1)}
  \end{align*}

  Since it is given that $\{a_n\}$ is increasing, we know that $a_{n + 1} \geq a_i \;\forall\; i \leq n$, and therefore, $a_{n + 1} - a_i \geq 0 \;\forall\; i \leq n \implies \sum_{i = 1}^{n}(a_{n + 1} - a_i) \geq 0$. $n > 0$, so the denominator is also positive, which means the difference between two consecutive terms is positive, that is, $\{b_n\}$ is increasing.
\end{proof}

\ppart{b}
\begin{proof}
  We are given that $\{a_n\}$ is bounded above, i.e. there exists $M \in \mathbb{R}$ such that $a_i \leq M \;\forall\; i$. Consider a term of the sequence $\{b_n\}$: $b_n = \frac{\sum_{i = 1}^n a_i}{n}$. Given the former condition, we know that $\sum_{i = 1}^n a_i \leq \sum_{i = 1}^n M = nM$. Thus, $b_n \leq \frac{nM}{n} = M \;\forall\; n$, i.e. $\{b_n\}$ is bounded above.
\end{proof}

\problem{2-2}
\begin{proof}
  Let $L$ be the smallest upper bound for $\{a_n\}$. Then, $a_i \leq L \;\forall\; i$, and there is no such number $M$ such that $M < L$ and  $a_i \leq M \;\forall\; i$. Since $\{a_n\}$ is increasing, this means that after a certain point, $a_i$ will get arbitrarily close to $L$ (since every term is greater than or equal to the previous term); that is, there exists $N$ such that if $n \geq N$ then for any $\epsilon > 0$, $L - a_n < \epsilon$.
\end{proof}

\end{document}
