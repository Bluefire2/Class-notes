\documentclass[12pt]{article}
\usepackage{fancyhdr}     % Enhanced control over headers and footers
\usepackage[T1]{fontenc}  % Font encoding
\usepackage{mathptmx}     % Choose Times font
\usepackage{microtype}    % Improves line breaks
\usepackage{setspace}     % Makes the document look like horse manure
\usepackage{lipsum}       % For dummy text
\usepackage{etoolbox}

\AtBeginEnvironment{quote}{\singlespacing\small}

\newcommand{\name}{Kirill Chernyshov}

\pagestyle{fancy} % Default page style
\lhead{\name}
\chead{}
\rhead{\thepage}
\lfoot{}
\cfoot{}
\rfoot{}
\renewcommand{\headrulewidth}{1pt}
\renewcommand{\footrulewidth}{1pt}

\thispagestyle{empty} %First page style

\setlength\headheight{15pt} %Slight increase to header size

\begin{document}
\begin{center}
\begin{tabular}{c}
\textbf{\name} \\
\textbf{\today}
\end{tabular}
\end{center}
\doublespacing

\begin{document}
\title{Math 3110 homework \#4}
\author{\name}
\maketitle

\problem{5.1.4}
Suppose $\frac{a_n}{b_n} \rightarrow L, b_n \neq 0 \;\forall\; n$ and $b_n \rightarrow 0$. Show that $a_n \rightarrow 0$.

\begin{proof}
  $a_n = \frac{a_n}{b_n} \cdot b_n$. Therefore, $\lim a_n = \lim \left(\frac{a_n}{b_n} \cdot b_n\right) = \lim \frac{a_n}{b_n} \cdot \lim b_n = L \cdot 0 = 0$, by the limit product rule.
\end{proof}

\problem{5.3.4}

\ppart{i}
\begin{proof}
  Since $b_n$ is convergent, it must also be bounded above. Since $a_n \leq b_n \;\forall\; n$, $a_n$ must also be bounded above. By a theorem from earlier in the course, since $a_n$ is also increasing, it must converge.

  Since both $a_n$ and $b_n$ are convergent sequences, $a_n \leq b_n \;\forall\; n$ implies that $\lim a_n \leq \lim b_n$, by the limit location theorem (part \textit{(15c)}).
\end{proof}

\ppart{ii}
A simple counterexample is the pair of sequences $a_n = \frac{1}{n^2}$ and $b_n = \frac{1}{n}$, for $n \geq 2$. Since $n \geq 2 \implies n^2 > n$, $a_n < b_n \;\forall\; n$. But, $\lim a_n = \lim b_n = 0$, so the statement $\lim a_n < \lim b_n$ is false.

\problem{5.4.1}

\ppart{a} \ppart{b}
Implied by part \textit{(c)}

\ppart{c}
\begin{proof}
  Suppose there are $K$ colourings, and denote the different subsequences as $a_k$, and their terms as $a_{k_m}$ for $0 < k \leq K, m \geq 1$. Suppose all of these subsequences converge to $L$, that is, $\;\forall\;k, \;\forall\; \epsilon > 0$ there exists $M_k$ such that $|a_{k_m} - L| < \epsilon \;\forall\; m \geq M_k$.

  Let $N = \max(\{M_i | 0 < i \leq K\})$. Then, one version of the above statement applies to all of the subsequences: $\;\forall\;k, \;\forall\; \epsilon > 0, |a_{k_m} - L| < \epsilon \;\forall\; m \geq N$.

  Since every term of $\{a_n\}$ is coloured in some way, we know that for any $a_n$ there exist $k, m$ such that $a_n = a_{k_m}$. Therefore, for any $\epsilon > 0$, $|a_n - L| = |a_{k_m} - L| < \epsilon \;\forall\; m \geq N$, that is, $\{a_n\}$ converges to $L$.
\end{proof}

\problem{5.4.2}

Suppose $s(n)$ is the sum of the prime factors of $n \in \mathbb{Z^+}$, and define the sequence $a_n = \frac{s(n)}{n}$. Show that $\lim_{n \rightarrow \infty} a_n$ does not exist.

\begin{proof}
  There are infinitely many numbers that are greater than the sum of their prime factors. An example is $10 > 5 + 2$. To show that there are infinitely many, note that if $n > s(n), n > 2$ then $s(2n) = s(n) + 2 < n + 2 < 2n$. There are also infinitely many prime numbers (\textit{Elements}, Euclid, c. 300BC), and if $n$ is prime, then $n = s(n)$.
  Define two subsequences of $\{a_n\}$ as follows: $\{b_k\}$ as the members of $\{a_n\}$ where $n$ was generated by the above rule, that is, $b_k = \frac{s(n)}{n}$ where $n = 10 \cdot 2^k$; and $\{c_p\}$ as $\frac{s(p)}{p}$ where $p$ is prime. In the case of $b_n$, since $n > s(n) \;\forall\; n$, $\frac{s(n)}{n} < 1 \;\forall\; n$. Consider the ratio between two consecutive terms:

  \begin{align*}
    \frac{b_{k + 1}}{b_k} &= \frac{s(10 \cdot 2^{k + 1})}{10 \cdot 2^{k + 1}} \cdot \frac{10 \cdot 2^k}{s(10 \cdot 2^k)} \\
    &= \frac{20(k + 1)}{10 \cdot 2^{k + 1}} \cdot \frac{10 \cdot 2^k}{20k} \\
    &= \frac{k + 1}{2k}
  \end{align*}

  Since for all $k \geq 2$, this ratio is less than $1$, this means that for $k \geq 2$, $\{b_k\}$ is decreasing, and so its limit must be less than one. On the other hand, $p_n = \frac{s(p)}{p} = 1$, so $\lim_{n \rightarrow \infty} p_n = 1$. Since the two subsequences converge to different limits, $a_n$ cannot have a limit.
\end{proof}

\problem{5-5}

Suppose $a_n \rightarrow L$, and $b_n$ lies between $a_n$ and $a_{n + 1}$ for all $n$. Show that $b_n \rightarrow L$.

\begin{proof}
  Define two subsequences of $b_n$: $\{b_{\alpha}\}$, consisting of terms of $\{b_n\}$ where $a_n \leq b_n \leq a_{n + 1}$, and $\{b_{\beta}\}$, consisting of the terms of $\{b_n\}$ for which $a_{n + 1} \leq b_n \leq a_n$. Since $a_n \rightarrow \infty$ and $a_{n + 1} \rightarrow \infty$, by the Squeeze theorem, $\{b_{\alpha}\}$ must converge to $L$, and $\{b_{\beta}\}$ must also converge to $L$. Since all terms of $\{b_n\}$ are in exactly one of $\{b_{\alpha}\}$ or $\{b_{\beta}\}$, $\{b_n\}$ converges to $L$ by the subsequence theorem proven in problem \textit{(5.4.1)}.
\end{proof}

\end{document}
