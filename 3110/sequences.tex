\documentclass[11pt]{amsart}

\usepackage{amssymb,amsmath}
\usepackage{bm}
\usepackage{mathtools}
\usepackage{framed}
\usepackage{esvect}
\usepackage{amsthm}
\usepackage{centernot}
\usepackage{ifxetex,ifluatex}

%%%%%%%%%%%%%%% FILL THIS IN FOR EACH ASSIGNMENT
\newcommand{\name}{Kyrylo Chernyshov}
\newcommand{\sectionnum}{203}
\newcommand{\norm}[1]{\left\lVert#1\right\rVert}
\newcommand*\diff{\mathop{}\!\mathrm{d}}
\newcommand*\Diff[1]{\mathop{}\!\mathrm{d^#1}}
\newcommand{\ex}{\subsubsection{Example}}
\newcommand*\mean[1]{\bar{#1}}

\DeclareMathOperator{\proj}{proj}
\DeclareMathOperator{\im}{im}
\DeclareMathOperator{\row}{row}
\DeclareMathOperator{\col}{col}
\DeclareMathOperator{\rank}{rank}
\DeclareMathOperator{\nullity}{nullity}
\DeclareMathOperator{\detm}{det}
\DeclarePairedDelimiter\ceil{\lceil}{\rceil}
\DeclarePairedDelimiter\floor{\lfloor}{\rfloor}
%%%%%%%%%%%%%%%%%%%%%%%%%%%%%%%%%%%%%%%%%%%%%%%%

\usepackage[margin=1in, letterpaper]{geometry}
\newcommand{\problem}[1]{\bigskip\noindent\textbf{Problem #1}}
\newcommand{\ppart}[1]{\bigskip\textbf{(#1)}}
\newcommand{\lemma}{\bigskip\textbf{Lemma}}
\newcommand{\E}{\mathrm{E}}
\newcommand{\Var}{\mathrm{Var}}
\newcommand{\Cov}{\mathrm{Cov}}

\begin{document}
\title{Math 3110 - Sequences}
\author{\name}
\maketitle

\section{Sequences}

\subsection{Definition}
A \textbf{sequence} is an ordered collection of rational or real numbers $\{a_n\}_{n = 1}^{\infty}$. Essentially, it is a map $a: \mathbb{N} \rightarrow \mathbb{Q}$ or $\mathbb{R}$, that is, $n \mapsto a_n$.

\ex
The set of numbers $1, \frac{1}{4}, \frac{1}{9}, \frac{1}{16} \cdots$, i.e. $a_n = \frac{1}{n^2}$ is a sequence.

\ex
Another example of a sequence is $1, 1 + \frac{1}{2}, 1 + \frac{1}{2} + \frac{1}{3} \cdots$, where $b_n = 1 + \frac{1}{2} + \frac{1}{3} + \cdots + \frac{1}{n} = \sum_{k = 1}^n \frac{1}{k}$.

Any sequence that can be expressed in the form

\begin{align*}
  b_n = \sum_{k = 1}^n s_k
\end{align*}

is called a \textbf{series}.

\subsection{Increasing sequences}
A sequence $\{a_n\}$ is \textbf{increasing} iff $a_{n + 1} \geq a_n \;\forall\; n$. Note that by this definition, $a_n = 1$ is an increasing sequence, since the inequality is not strict. Therefore, we can also define a sequence as \textbf{strictly increasing} iff $a_{n + 1} > a_n \;\forall\; n$. An example of a strictly increasing sequence is the sequence $b_n = \sum_{k = 1}^n \frac{1}{k}$ from above, since $b_{n + 1} - b_n = \frac{1}{n + 1} > 0$. We can also define \textbf{decreasing} and \textbf{strictly decreasing} sequences in a similar way. A sequence that is either increasing \textbf{or} decreasing is called a \textbf{monotonic} sequence.

\subsection{Bounded sequences}
A sequence $\{a_n\}$ is \textbf{bounded above} iff there exists $M \in \mathbb{R}$ such that $a_n \leq M \;\forall\; n$. An example of a bounded sequence is $a_n = \frac{1}{n^2}$ from earlier, since $a_n \leq 1 \;\forall\; n$. Note that $M$ does not necessarily have to be the \textit{best} upper bound, but simply \textit{an} upper bound; for example, setting $M = 100$ would also be sufficient to show that this sequence is bounded above.

Similarly, a sequence $\{a_n\}$ is \textbf{bounded below} iff there exists $P \in \mathbb{R}$ such that  $a_n \geq P \;\forall\; n$. Finally, a sequence is \textbf{bounded above and below} iff it is bounded above and bounded below.

\ex
Once again examine the sequence $b_n = \sum_{k = 1}^n \frac{1}{k}$. This sequence is bounded below, but not bounded above

\begin{proof}
  Trivially, $b_1 \geq 1$. We know that the sequence is increasing, that is, $b_{n + 1} > b_n \;\forall\; n$. Therefore, $b_n > b_1 \geq 1$, so the sequence is bounded below with $P = 1$.

  Proving that it is not bounded above is harder. Take any $M \in \mathbb{R}$. It can be seen by analysing the series that for $k > 2M$, $b_{2^k} > 1 + \frac{k}{2} > 1 + \frac{2M}{2} = 1 + M > M$. Therefore, the sequence is not bounded above by any $M$.
\end{proof}

\section{Convergence (an imprecise definition)}
We say a sequence $\{a_n\}$ \textbf{converges to $L$} iff for all integers $k \geq 0$, there exists an integer $N$, such that for all $n \geq N$, $a_n$ and $L$ are equal up to $k$ decimal places.

\ex
Consider the sequence $a_1 = 1, a_2 = 1.1, a_3 = 1.11 \cdots $. This sequence is bounded by $L = 1.\overline{1} = \frac{10}{9}$.

\begin{proof}

\end{proof}

\end{document}
