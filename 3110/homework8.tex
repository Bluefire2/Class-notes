\documentclass[11pt]{amsart}

\usepackage{amssymb,amsmath}
\usepackage{bm}
\usepackage{mathtools}
\usepackage{framed}
\usepackage{esvect}
\usepackage{amsthm}
\usepackage{centernot}
\usepackage{ifxetex,ifluatex}

%%%%%%%%%%%%%%% FILL THIS IN FOR EACH ASSIGNMENT
\newcommand{\name}{Kyrylo Chernyshov}
\newcommand{\sectionnum}{203}
\newcommand{\norm}[1]{\left\lVert#1\right\rVert}
\newcommand*\diff{\mathop{}\!\mathrm{d}}
\newcommand*\Diff[1]{\mathop{}\!\mathrm{d^#1}}
\newcommand{\ex}{\subsubsection{Example}}
\newcommand*\mean[1]{\bar{#1}}

\DeclareMathOperator{\proj}{proj}
\DeclareMathOperator{\im}{im}
\DeclareMathOperator{\row}{row}
\DeclareMathOperator{\col}{col}
\DeclareMathOperator{\rank}{rank}
\DeclareMathOperator{\nullity}{nullity}
\DeclareMathOperator{\detm}{det}
\DeclarePairedDelimiter\ceil{\lceil}{\rceil}
\DeclarePairedDelimiter\floor{\lfloor}{\rfloor}
%%%%%%%%%%%%%%%%%%%%%%%%%%%%%%%%%%%%%%%%%%%%%%%%

\usepackage[margin=1in, letterpaper]{geometry}
\newcommand{\problem}[1]{\bigskip\noindent\textbf{Problem #1}}
\newcommand{\ppart}[1]{\bigskip\textbf{(#1)}}
\newcommand{\lemma}{\bigskip\textbf{Lemma}}
\newcommand{\E}{\mathrm{E}}
\newcommand{\Var}{\mathrm{Var}}
\newcommand{\Cov}{\mathrm{Cov}}

\begin{document}
\title{Math 3110 homework \#8}
\author{\name}
\maketitle

\problem{1}
\begin{proof}
  Construct two sequences: $\{a_n\}: a_n = \frac{\pi}{2} + 2n\pi$, and $\{b_n\}: b_n = 2n\pi$. Clearly, $a_n \rightarrow \infty$ and $b_n \rightarrow \infty$. Suppose there exists $L \in \mathbb{R}$ such that $\lim_{x \rightarrow \infty} \sin x = L$. Then, by the limit form of sequential continuity, $\lim_{n \rightarrow \infty} \sin a_n = \lim_{n \rightarrow \infty} \sin b_n = L$. But $\sin a_n = 1, \sin b_n = -1 \;\forall\; n \implies 1 = -1$. Therefore, by contradiction, $\lim_{x \rightarrow \infty} \sin x$ cannot exist.
\end{proof}

\problem{2}
\begin{proof}
  Suppose that there exists a continuous $f$ such that $f(x) \in \mathbb{Q} \;\forall\; x$. Pick $a, b \in \mathbb{R}, f(a) \neq f(b)$; then we have $f(a), f(b) \in \mathbb{Q}$. By a fundamental theorem, there exists $x: f(a) < x < f(b)$ such that $x \notin \mathbb{Q}$. But by the intermediate value theorem, there must exist $c: a < c < b$ such that $f(c) = x$. Therefore, the condition $f(a) \neq f(b)$ cannot hold, i.e. the value of $f(x)$ is constant for all $x$.
\end{proof}

\problem{3}
\begin{proof}
  Let $f(x) = y^2 \cos x - e^x$ for some $y \geq 1$. $f(0) = y^2 - 1 \geq 0$, and $f\left(\frac{\pi}{2}\right) = -e^{\frac{\pi}{2}} < 0$. Since $f$ changes sign on $\left[0, \frac{\pi}{2}\right]$, by Bolzano's theorem, it must have a zero in $\left[0, \frac{\pi}{2}\right]$. This zero cannot be $\frac{\pi}{2}$, since as above, $f\left(\frac{\pi}{2}\right) < 0$, so the zero must be in $\left[0, \frac{\pi}{2}\right)$.
\end{proof}

\problem{4}

$k(-4) = -19$, $k(-2) = 13$, $k(0) = -3$, $k(2) = -19$ and $k(4) = 13$. There are, then, at least three intervals on which $k$ changes sign: $[-4, -2]$, $[-2, 0]$ and $[2, 4]$. $k$ is continuous, since it is a linear combination of continuous functions. Therefore, by Bolzano's theorem, it has roots in all three of these intervals. Since $k(-2) \neq 0$, $-2$ is not a root, so the roots in the intervals $[-4, -2]$ and $[-2, 0]$ must be distinct (they cannot both be $-2$), and thus $k(x) = 0$ has at least three roots. But since $k(x) = 0$ is cubic, it can have at most three roots; therefore, $k(x) = 0$ has exactly 3 roots.

\problem{5}

\ppart{a}
E

\ppart{b}
C

\ppart{c}
$[A, C]$ and $[E, I]$, because the function is continuous on both these intervals.

\ppart{d}
Same as part (c), and $[A, F]$, since on $[A, F]$ $p(x)$ takes all values between $p(A)$ and $p(F)$.

\end{document}
