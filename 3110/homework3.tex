\documentclass[11pt]{amsart}

\usepackage{amssymb,amsmath}
\usepackage{bm}
\usepackage{mathtools}
\usepackage{framed}
\usepackage{esvect}
\usepackage{amsthm}
\usepackage{centernot}
\usepackage{ifxetex,ifluatex}

%%%%%%%%%%%%%%% FILL THIS IN FOR EACH ASSIGNMENT
\newcommand{\name}{Kyrylo Chernyshov}
\newcommand{\sectionnum}{203}
\newcommand{\norm}[1]{\left\lVert#1\right\rVert}
\newcommand*\diff{\mathop{}\!\mathrm{d}}
\newcommand*\Diff[1]{\mathop{}\!\mathrm{d^#1}}
\newcommand{\ex}{\subsubsection{Example}}
\newcommand*\mean[1]{\bar{#1}}

\DeclareMathOperator{\proj}{proj}
\DeclareMathOperator{\im}{im}
\DeclareMathOperator{\row}{row}
\DeclareMathOperator{\col}{col}
\DeclareMathOperator{\rank}{rank}
\DeclareMathOperator{\nullity}{nullity}
\DeclareMathOperator{\detm}{det}
\DeclarePairedDelimiter\ceil{\lceil}{\rceil}
\DeclarePairedDelimiter\floor{\lfloor}{\rfloor}
%%%%%%%%%%%%%%%%%%%%%%%%%%%%%%%%%%%%%%%%%%%%%%%%

\usepackage[margin=1in, letterpaper]{geometry}
\newcommand{\problem}[1]{\bigskip\noindent\textbf{Problem #1}}
\newcommand{\ppart}[1]{\bigskip\textbf{(#1)}}
\newcommand{\lemma}{\bigskip\textbf{Lemma}}
\newcommand{\E}{\mathrm{E}}
\newcommand{\Var}{\mathrm{Var}}
\newcommand{\Cov}{\mathrm{Cov}}

\begin{document}
\title{Math 3110 homework \#3}
\author{\name}
\maketitle

\problem{3.1.1}
\ppart{e}

\begin{proof}
  We begin by examining the difference $|\sqrt{n^2 + 1} - n - 0| = |\sqrt{n^2 + 1} - n|$. Since for $n > 0$, $\sqrt{n^2} = n$ and $n^2 + 1 > n^2$, and since $\sqrt{n}$ increases as $n$ increases, we have $\sqrt{n^2 + 1} > n$ for $n > 0$. Therefore, $|\sqrt{n^2 + 1} - n| = \sqrt{n^2 + 1} - n$, since the quantity is always positive. Now, pick $\epsilon > 0$ and solve the inequality:

  \begin{align*}
    \sqrt{n^1 + 1} - n &< \epsilon \\
    \sqrt{n^2 + 1} &< \epsilon + n \\
    n^2 + 1 &< \epsilon^2 + n^2 + \epsilon n \\
    \epsilon n &> 1 - \epsilon^2 \\
    n &> \frac{1}{\epsilon} - \epsilon
  \end{align*}

  Therefore, given $N > \frac{1}{\epsilon} - \epsilon$, for any $\epsilon$, for all $n \geq N$, $|\sqrt{n^2 + 1} - n| < \epsilon$.
\end{proof}

\problem{3.1.2}
Prove that if $\{a_n\}$ is a non-negative sequence, then $\lim_{n \rightarrow \infty} a_n = 0 \implies \lim_{n \rightarrow \infty} \sqrt{a_n} = 0$.

\begin{proof}
  Pick any $\epsilon > 0$. By definition, $\lim_{n \rightarrow \infty} a_n = 0$ means that, for any $\epsilon^{\prime} > 0$ there exists $N$ such that for all $n \geq N$, $|a_n| < \epsilon^{\prime}$, equivalent to $a_n < \epsilon^{\prime}$ since we are also given that $a_n \geq 0 \;\forall\; n$. $\sqrt{n}$ increases as $n$ increases, so this means that $\sqrt{a_n} < \sqrt{\epsilon^{\prime}}$. Therefore, we can pick $\epsilon^{\prime} = \epsilon^2$, and using the same $N$, we know that $\sqrt{a_n} < \sqrt{\epsilon^{\prime}} = \epsilon$. Therefore, $\lim_{n \rightarrow \infty} \sqrt{a_n} = 0$.
\end{proof}

\problem{3.2.3}

\ppart{a}
Show that the sequence

\begin{align*}
  a_n = \sum_{i = 1}^{n} \frac{1}{n + i}
\end{align*}

has a limit.

\begin{proof}
  To show that this sequence has a limit, it suffices to show that it is both increasing and has an upper bound. First, examine the difference between two consecutive terms:

  \begin{align*}
    a_{n + 1} - a_n &= \sum_{i = 1}^{n + 1} \frac{1}{n + 1 + i} - \sum_{i = 1}^{n} \frac{1}{n + i} \\
    &= \sum_{i = 2}^{n + 2} \frac{1}{n + i} - \sum_{i = 1}^{n} \frac{1}{n + i} \\
    &= \frac{1}{2n + 2} + \frac{1}{2n + 1} - \frac{1}{n + 1} \\
    &> \frac{1}{2n + 2} + \frac{1}{2n + 2} - \frac{1}{n + 1} = \frac{1}{n + 1} - \frac{1}{n + 1} = 0
  \end{align*}

  Since the difference is strictly greater than $0$, the sequence is strictly increasing. Also, note that $\sum_{i = 1}^{n} \frac{1}{n + i} < \sum_{i = 1}^{n} \frac{1}{n} = \frac{n}{n} = 1$, that is, the sequence is bounded above by $1$. Thus, the sequence must have a limit.
\end{proof}

The limit is actually $\ln 2$, since the sequence is actually the riemann sum for $\int_0^1 \frac{1}{1 + x} \diff x$

\ppart{b}
In the last step, the $K-\epsilon$ principle cannot be applied, since it works if and only if $K$ is a constant, and $n$ is not a constant.

\problem{4.3.1}
The generic recursive formula for Newton's method is

\begin{align*}
  a_{n + 1} = a_n - \frac{f(a_n)}{f^{\prime}(x_n)}
\end{align*}

For $f(x) = x^2 - 3$ we have $f^{\prime}(x) = 2x$, and therefore the recursive formula is $a_{n + 1} = a_n - \frac{a_n^2 - 3}{2a_n}$, where $a_n \neq 0$.

\problem{3-1}
\ppart{a}
Suppose that $b_n = \frac{\sum_{i = 1}^n a_i}{n}$ for some sequence $\{a_n\}$. Show that if $\lim_{n \rightarrow \infty} a_n = 0$ then $\lim_{n \rightarrow \infty} b_n = 0$.

\begin{proof}
  By definition, given $\epsilon > 0$, there exists $N$ such that for all $n \geq N$, $|a_n| < \epsilon$. Now examine the value of $|b_n|$:

  \begin{align*}
    |b_n| &= \Big|\frac{\sum_{i = 1}^n a_i}{n}\Big| \\
    &= \Big|\frac{\sum_{i = 1}^{\floor{N}} a_i}{n} + \frac{\sum_{i = \floor{N} + 1}^n a_i}{n}\Big| \\
    &< \frac{\sum_{i = 1}^{\floor{N}} |a_i|}{n} + \frac{\sum_{i = \floor{N} + 1}^n \epsilon}{n} \\
    &= \frac{\sum_{i = 1}^{\floor{N}} |a_i|}{n} + \frac{(n - \floor{N})\epsilon}{n} \\
    &< \frac{\sum_{i = 1}^{\floor{N}} a_i}{n} + \frac{n\epsilon}{n} \\
    &= \frac{\sum_{i = 1}^{\floor{N}} a_i}{n} + \epsilon \\
  \end{align*}

  Since $a_n$ must be bounded, we know that there exists $k$ such that for all $n$, $|a_n| \leq k$. This means that $\frac{\sum_{i = 1}^{\floor{N}} a_i}{n} \leq \frac{\sum_{i = 1}^{\floor{N}} k}{n} = \frac{k \cdot \floor{N}}{n}$. If we pick $n$ such that $n \geq \max(N, \frac{k \cdot \floor{N}}{\epsilon})$, then $|b_n| < \frac{\sum_{i = 1}^{\floor{N}} a_i}{n} + \epsilon \leq \epsilon + \epsilon = 2\epsilon$. Therefore, by the $K-\epsilon$ principle, $\lim_{n \rightarrow \infty} b_n = 0$.
\end{proof}

\problem{3-4}
Prove that a convergent sequence $\{a_n\}$ must be bounded.

\begin{proof}
  Suppose $\lim_{n \rightarrow \infty} a_n = L$. Then, by definition, for any $\epsilon > 0$, there exists $N$ such that for all $n \geq N$, $|a_n - L| < \epsilon$. For any such $\epsilon$, we know that the sequence $\{a_n\}$ where $n \geq N$ must be bounded, since $|a_n - L| < \epsilon \implies L - \epsilon < a_n < L + \epsilon$. The rest of the sequence (the terms $a_0, a_1 \cdots a_{\floor{N}}$) is a finite set of finite terms, therefore it must also be bounded by some constants $P$ and $M$, such that for all $i: 0 < i \leq \floor{N}$, $P \leq a_i \leq M$. Therefore, we can construct a lower and an upper bound on the entire sequence:

  \begin{align*}
    \min(P, L - \epsilon) \leq a_n \leq \max(M, L + \epsilon) \;\forall\; n
  \end{align*}
\end{proof}

\end{document}
