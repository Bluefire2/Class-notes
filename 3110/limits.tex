\documentclass[11pt]{amsart}

\usepackage{amssymb,amsmath}
\usepackage{bm}
\usepackage{mathtools}
\usepackage{framed}
\usepackage{esvect}
\usepackage{amsthm}
\usepackage{centernot}
\usepackage{ifxetex,ifluatex}

%%%%%%%%%%%%%%% FILL THIS IN FOR EACH ASSIGNMENT
\newcommand{\name}{Kyrylo Chernyshov}
\newcommand{\sectionnum}{203}
\newcommand{\norm}[1]{\left\lVert#1\right\rVert}
\newcommand*\diff{\mathop{}\!\mathrm{d}}
\newcommand*\Diff[1]{\mathop{}\!\mathrm{d^#1}}
\newcommand{\ex}{\subsubsection{Example}}
\newcommand*\mean[1]{\bar{#1}}

\DeclareMathOperator{\proj}{proj}
\DeclareMathOperator{\im}{im}
\DeclareMathOperator{\row}{row}
\DeclareMathOperator{\col}{col}
\DeclareMathOperator{\rank}{rank}
\DeclareMathOperator{\nullity}{nullity}
\DeclareMathOperator{\detm}{det}
\DeclarePairedDelimiter\ceil{\lceil}{\rceil}
\DeclarePairedDelimiter\floor{\lfloor}{\rfloor}
%%%%%%%%%%%%%%%%%%%%%%%%%%%%%%%%%%%%%%%%%%%%%%%%

\usepackage[margin=1in, letterpaper]{geometry}
\newcommand{\problem}[1]{\bigskip\noindent\textbf{Problem #1}}
\newcommand{\ppart}[1]{\bigskip\textbf{(#1)}}
\newcommand{\lemma}{\bigskip\textbf{Lemma}}
\newcommand{\E}{\mathrm{E}}
\newcommand{\Var}{\mathrm{Var}}
\newcommand{\Cov}{\mathrm{Cov}}

\begin{document}
\title{Math 3110 - Limits}
\author{\name}
\maketitle

\section{Limits}
\subsection{Definition}
We define an infinite limit of a sequence as follows:

$\lim_{n \rightarrow \infty} a_n = L$ means that $\forall\; \epsilon > 0$, $\exists N \in \mathbb{R}$ such that $\forall\; n \geq N$, $|a_n - L| < \epsilon$.

\subsection{Uniqueness of limits}
Suppose $\lim_{n \rightarrow \infty} a_n = L$ and $\lim_{n \rightarrow \infty} a_n = M$. Then, $L = M$.

\begin{proof}
  Suppose that instead, $L \neq M$. Let $\epsilon^{\prime} = \frac{|L - M|}{2} > 0$. $\lim_{n \rightarrow \infty} a_n = L$ means that $|a_n - L| < \epsilon^{\prime}$ for all $n \geq N_1$. Similarly, $\lim_{n \rightarrow \infty} a_n = M$ means that $|a_n - M| < \epsilon^{\prime}$ for all $n \geq N_2$. Combining these inequalities, we can say that $-\epsilon^{\prime} < a_n - L < \epsilon^{\prime}$ \textbf{and} $-\epsilon^{\prime} < a_n - M < \epsilon^{\prime}$ for all $n \geq \mathrm{max}(N_1, N_2)$. Subtracting, we get the largest and smallest values of the difference between the two limits: $-2\epsilon^{\prime} < M - L < 2\epsilon^{\prime}$, that is $|M - L| < 2\epsilon^{\prime}$.

  But, we defined $\epsilon^{\prime}$ as $\frac{|M - L|}{2}$. Substituting, we get the inequality $|M - L| < |M - L|$, which obviously cannot hold. Therefore, the initial premise that $\frac{|L - M|}{2} > 0$, that is, $L \neq M$, cannot be true.
\end{proof}


\end{document}
