\documentclass[12pt]{article}
\usepackage{fancyhdr}     % Enhanced control over headers and footers
\usepackage[T1]{fontenc}  % Font encoding
\usepackage{mathptmx}     % Choose Times font
\usepackage{microtype}    % Improves line breaks
\usepackage{setspace}     % Makes the document look like horse manure
\usepackage{lipsum}       % For dummy text
\usepackage{etoolbox}

\AtBeginEnvironment{quote}{\singlespacing\small}

\newcommand{\name}{Kirill Chernyshov}

\pagestyle{fancy} % Default page style
\lhead{\name}
\chead{}
\rhead{\thepage}
\lfoot{}
\cfoot{}
\rfoot{}
\renewcommand{\headrulewidth}{1pt}
\renewcommand{\footrulewidth}{1pt}

\thispagestyle{empty} %First page style

\setlength\headheight{15pt} %Slight increase to header size

\begin{document}
\begin{center}
\begin{tabular}{c}
\textbf{\name} \\
\textbf{\today}
\end{tabular}
\end{center}
\doublespacing

\begin{document}
\title{Math 3110 - Limits}
\author{\name}
\maketitle

\section{Limits}
\subsection{Definition}
We define an infinite limit of a sequence as follows:

$\lim_{n \rightarrow \infty} a_n = L$ means that $\forall\; \epsilon > 0$, $\exists N \in \mathbb{R}$ such that $\forall\; n \geq N$, $|a_n - L| < \epsilon$.

\subsection{Uniqueness of limits}
Suppose $\lim_{n \rightarrow \infty} a_n = L$ and $\lim_{n \rightarrow \infty} a_n = M$. Then, $L = M$.

\begin{proof}
  Suppose that instead, $L \neq M$. Let $\epsilon^{\prime} = \frac{|L - M|}{2} > 0$. $\lim_{n \rightarrow \infty} a_n = L$ means that $|a_n - L| < \epsilon^{\prime}$ for all $n \geq N_1$. Similarly, $\lim_{n \rightarrow \infty} a_n = M$ means that $|a_n - M| < \epsilon^{\prime}$ for all $n \geq N_2$. Combining these inequalities, we can say that $-\epsilon^{\prime} < a_n - L < \epsilon^{\prime}$ \textbf{and} $-\epsilon^{\prime} < a_n - M < \epsilon^{\prime}$ for all $n \geq \mathrm{max}(N_1, N_2)$. Subtracting, we get the largest and smallest values of the difference between the two limits: $-2\epsilon^{\prime} < M - L < 2\epsilon^{\prime}$, that is $|M - L| < 2\epsilon^{\prime}$.

  But, we defined $\epsilon^{\prime}$ as $\frac{|M - L|}{2}$. Substituting, we get the inequality $|M - L| < |M - L|$, which obviously cannot hold. Therefore, the initial premise that $\frac{|L - M|}{2} > 0$, that is, $L \neq M$, cannot be true.
\end{proof}


\end{document}
