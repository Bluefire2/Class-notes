\documentclass[12pt]{article}
\usepackage{fancyhdr}     % Enhanced control over headers and footers
\usepackage[T1]{fontenc}  % Font encoding
\usepackage{mathptmx}     % Choose Times font
\usepackage{microtype}    % Improves line breaks
\usepackage{setspace}     % Makes the document look like horse manure
\usepackage{lipsum}       % For dummy text
\usepackage{etoolbox}

\AtBeginEnvironment{quote}{\singlespacing\small}

\newcommand{\name}{Kirill Chernyshov}

\pagestyle{fancy} % Default page style
\lhead{\name}
\chead{}
\rhead{\thepage}
\lfoot{}
\cfoot{}
\rfoot{}
\renewcommand{\headrulewidth}{1pt}
\renewcommand{\footrulewidth}{1pt}

\thispagestyle{empty} %First page style

\setlength\headheight{15pt} %Slight increase to header size

\begin{document}
\begin{center}
\begin{tabular}{c}
\textbf{\name} \\
\textbf{\today}
\end{tabular}
\end{center}
\doublespacing

\begin{document}
\title{Medical Microbiology Tutorial \#2: AMP Resistant Bacteria}
\author{Charlie Cranston}
\maketitle

\problem{1}
AMP action mechanisms include the disruption of cell wall synthesis, like β-lactams (e.g. penicillin) which prevent the action of transpeptidases forming cross-bridges in peptidoglycan. Other enzymes can be targeted by different AMPs to kill bacterial cells. This means that the cell has reduced structure rigidity and so can swell and burst. AMPs can act upon 70S ribosomes to prevent protein synthesis and so cell function breaks down, meaning the cell will die. An example of this AMP is Bac71-35 which acts upon E. coli (Mardirossian et al., 2014 \cite{mardirossian}). There is an increase of interest with AMPs that are cationic, meaning they are positively charged, due to the fact that they have shown potent antibiotic ability. Therefore, they are attracted to negatively charged molecules like LPS, specifically the phosphate groups that are associated with LPS. The action of binding to disrupts the membrane and LPS and contents can leak out.
But some bacteria show resistance by removing a phosphate off lipid A and so there is a smaller negative charge on LPS which means a weaker interaction between LPS and cationic AMPs. There are other mechanisms which involve the use of membrane efflux pumps. Pumps are mainly driven by the proton motive force and can recognise specific substrates or a larger group. For example, there is a microlide pump in Group A Streptococcus and there is a multidrug pump in P. aeruginosa (Nizet, 2006 \cite{nizetv}). Another mechanism involves the external trapping of AMPs by neutralising molecules, complementary to the AMP. This can be done directly from the bacterial cell, or indirectly by forcing intracellular secretion of a binding molecule in a host cell, like S. aureus producing staphylokinase to cause plasminogen secretion in a host cell which neutralises α-defensins can also be degraded by peptidases. But trapping molecules can be produced by degrading host cell processes. Decorin and other host cell surface proteoglycans can be degraded by bacteria (e.g. S. aureus, P. aeruginosa) into dermatan sulphate. Free dermatan sulphate can then bind to and neutralise α-defensins (Nizet, 2006 \cite{nizetv}). Peptidases can also be recruited to degrade the AMPs e.g. gelatinase from E. faecalis and elastase from P. aeruginosa degrade the cathelicidin, LL-37 by cleaving at Asn-Leu and Asp-Phe bonds (Nizet, 2006 \cite{nizetv}).

\problem{2}
For polymyxin B (PMB) and colistin (COL), MICs were obtained by putting 200µl of 0.1a.u. absorbance at 600nm (A600) of liquid cultured bacterial strains on separate TYGS agar plates and then being allowed to cool for 10 minutes before an E-strip (from Biomerieux) was placed on each of the agar plates. The plates were then left to incubate for 48 hours at 37. The E-strips were interpreted using the manufacturer’s instructions. The experiment was repeated twice more and an average MIC, as well as a standard deviation was calculated for each compound.
For human cathelicidin LL-37 (hLL37) from AnaSpec, murine cathelicidin-related AMP (mCRAMP) from Peptide Institute and human α-defensin 5 (hAD5), also from AnaSpec, the MICs were obtained by using the Hancock Laboratory Broth Microtiter Dilution Method, albeit with some slight modifications. Each strain was initially grown on an TYGS agar plate for roughly 48 hours at 37. Then, a 100µl sample containing 0.05a.u. A600 of each strain, as well as a chosen AMP was added into a well in a 96-well microtiter plate. The plates were then incubated for 48 hours, with the A600 under continuous observation using Tecan’s microplate reader. A positive growth control with no AMP and a negative growth control containing no bacteria was done as well. Again, each experiment was repeated twice after, an average MIC and standard deviation was calculated for each strain-AMP combination. MICs were made compared to the A600 during late log phase growth of the positive control, in that the MIC was to be when growth was reduced to below 50\% the growth of the positive control.
The results show that for all of the AMPs tested, the commensal bacteria generally had much higher MICs compared to the enteropathogen/E. coli bacteria. This is likely due to the fact that the most commensal strains have the LpxF gene to increase AMP resistance. However, this isn’t the most conclusive evidence as many commensal bacteria were tested and only a few enteropathogens were tested.

\problem{3}
Germ-free mice were kept in flexible plastic, gnotobiotic containers on a 12 hour light/dark cycle. They were between 8 and 12 weeks old and were all given (ad libitum) the same mouse chow, which was autoclaved. 20 mice were then inoculated with a 1 x 108 CFU mixture with a ratio of 1:1:1 B. thetaiotaomicron wild-type, lxpF deletant mutant and lxpF complemented mutant and the mice were divided into four groups where they were either given 1 x 108 CFU of C. rodentium wild type or the tir mutant type at day 7 by oral gavage, or they were given (ad libitum) 3\% dextran sulphate solution (DSS) in their drinking water from day 7 to 14, or they were left untreated. The mice were then sacrificed on day 28. Throughout the experiment at certain time points, the mice’s faecal DNA was purified and multiplied by qPCR methods, which showed the abundance of each bacterial strain.
What the results showed was that when there was host inflammation, (which meant the release of AMPs) like it was with the infected C. rodentium mice and the DSS treated mice, the lxpF deletant mutant strain was easily outcompeted, but when there was no host inflammation, as it was with mice colonised by the tir mutant of C. rodentium and with the mice which were left untreated, all bacterial strains stayed relatively stable in numbers. This indicates that the LxpF gene is responsible for AMP resistance. Also, when the bacteria were exposed to other AMPs in vitro, the results showed the same effect as with the in vivo model.

\problem{4}
\ppart{a}
Fig. 4A and 4B show that before C. rodentium infection, roughly 30\% is made up of wild-type B. thetaiotaomicron and then when there is an initial infection, the whole community is relatively stable and the percentage composition of the B. thetaiotaomicron wild type species is a bit below 30\% before increasing to about 50\% by the end of the infection. However, for the lxpF deletant mutant, the levels of the bacteria are at about 25\% and by the end of the infection from C. rodentium, there is no mutant bacteria remaining and the percentage composition of B. uniformis increases from about 15\% before the infection to about 40\% after the infection. This means that the lxpF mutant bacteria are not resilient to the AMPs associated with host inflammation, and so is eradicated whereas the other bacteria in the community are wild type commensals and so have AMP resistance. Then, upon further investigation with mice that have a fully mature innate immune system and a complete microbiota, they too show the same results of the previous experiment which has a part of the complete human microbiota, meaning that the mice do not have a fully mature innate immune system. Fig. 4C and 4E show the difference between the levels of wild type and lxpF deletant mutant when SPF Rag-/- mice who have been infected with C. rodentium and when the same mice have undergone no treatment. The figures show that when untreated, the relative levels of each strain remain similar at 0.06 relative units throughout the 28 days of observation. However, when infected at day 7, the levels of the lxpF mutant strain quickly drop off to 0 but the levels of the wild type strain remain very similar to how they were in the untreated mice. Fig. 4E provides additional evidence to show that most of the human commensals tested from human subjects (about 50\% of all genus in human microbiota) have PMB resistance, meaning that is it likely they have managed to survive in the human gut because they have the LxpF gene.

\ppart{b}
These findings suggest that there are large selection pressures upon the commensals of the host gut when there is host inflammation; more so than the effects from enteropathenogenic infections. It also shows that there has been evolution within the commensals so that they display a certain amount of AMP resistance and the host has a certain tolerance, which might be aided by the modification of Lipid A through LxpF, so that the bacteria can survive through a large range of disturbances, whether it be hormonal changes, external infections, autoimmune diseases etc., but without leading to bacterial depletion or overgrowth as either of these can be very bad for the host.

% References
\begin{thebibliography}{9}
\bibitem{mardirossian}
Mardirossian M., Grzela R., Giglione C., et al., 2014.
\textit{The Host Antimicrobial Peptide Bac71-35 binds to bacterial ribosomal proteins and inhibits protein synthesis}.
Chem. Biol. 21(12): p1639-47

\bibitem{nizetv}
Nizet V., 2006.
\textit{Antimicrobial Peptide Resistance Mechanisms of Human Bacterial Pathogens}.
Curr. Issues Mol. Biol. 8: p11–26.
\end{thebibliography}
\end{document}
