\documentclass[11pt]{amsart}

\usepackage{amssymb,amsmath}
\usepackage{bm}
\usepackage{mathtools}
\usepackage{framed}
\usepackage{esvect}
\usepackage{amsthm}
\usepackage{centernot}
\usepackage{ifxetex,ifluatex}

%%%%%%%%%%%%%%% FILL THIS IN FOR EACH ASSIGNMENT
\newcommand{\name}{Kyrylo Chernyshov}
\newcommand{\sectionnum}{203}
\newcommand{\norm}[1]{\left\lVert#1\right\rVert}
\newcommand*\diff{\mathop{}\!\mathrm{d}}
\newcommand*\Diff[1]{\mathop{}\!\mathrm{d^#1}}
\newcommand{\ex}{\subsubsection{Example}}
\newcommand*\mean[1]{\bar{#1}}

\DeclareMathOperator{\proj}{proj}
\DeclareMathOperator{\im}{im}
\DeclareMathOperator{\row}{row}
\DeclareMathOperator{\col}{col}
\DeclareMathOperator{\rank}{rank}
\DeclareMathOperator{\nullity}{nullity}
\DeclareMathOperator{\detm}{det}
\DeclarePairedDelimiter\ceil{\lceil}{\rceil}
\DeclarePairedDelimiter\floor{\lfloor}{\rfloor}
%%%%%%%%%%%%%%%%%%%%%%%%%%%%%%%%%%%%%%%%%%%%%%%%

\usepackage[margin=1in, letterpaper]{geometry}
\newcommand{\problem}[1]{\bigskip\noindent\textbf{Problem #1}}
\newcommand{\ppart}[1]{\bigskip\textbf{(#1)}}
\newcommand{\lemma}{\bigskip\textbf{Lemma}}
\newcommand{\E}{\mathrm{E}}
\newcommand{\Var}{\mathrm{Var}}
\newcommand{\Cov}{\mathrm{Cov}}

\begin{document}
\title{The variance of a random variable}
\maketitle

\section{Motivation}
Let $X \thicksim f(x)$ where

\begin{align*}
  f(x) =
  \begin{cases}
    2x & 0 \leq x \leq 1 \\
    0 & \text{otherwise}
  \end{cases}
\end{align*}

Then, we have $\E(X) = \int_0^1 2x^2 \diff x = \frac{2}{3}$. Now, let $Y \thicksim g(y) = \frac{2}{3} \;\forall\; y \in \mathbb{R}$. $\E(X) = \E(Y)$, but clearly $X$ and $Y$ are very different, since $X$ can vary whereas $Y$ can only take on its expected value, and therefore does so with 100\% probability. This demonstrates that the expected value \textbf{alone} is not a good way to describe a random variable.

\section{Defining variance}
We want to have a measure of how much a random variable $X$ can deviate from its expected value. We can define this deviation as $X - \E(X)$. However, it doesn't matter in which direction this difference is; that is, we want to treat positive and negative values of this the same. We can fix this by instead looking at $[X - \E(X)]^2$. We want to measure what value this is expected to take, that is, its \textit{expected value}. With this in mind, let us define

\begin{align*}
  \Var(X) &= \E([X - \E(X)]^2) \\
  &= \E(X^2 + \E(X)^2 - 2X\E(X)) \\
  &= \E(X^2) + \E(X)^2 - \E(2X\E(X)) \\
  &= \E(X^2) + \E(X)^2 - 2\E(X)^2 \\
  &= \E(X^2) - \E(X)^2
\end{align*}

\subsection{Calculating variance}
To calculate this we can just apply the formula for the expected value of a variable. Let $\mu = \E(X)$. Then, for a discrete random variable,

\begin{align*}
  \sum_x x^2 f(x) - \mu^2
\end{align*}

and for a continuous random variable,

\begin{align*}
  \Var(X) = \int_{-\infty}^{\infty} x^2 f(x) \diff x - \mu^2
\end{align*}

\subsection{The standard deviation}
Since the variance is defined as the expected value of the \textit{square} of the deviation, its units are the units of the random variable squared. For practical purposes, we may want a measure that comes in the same units as our random variable. For this we can define the \textbf{standard deviation} $\sigma$ as $\sqrt{\Var(X)}$. This is valid since the variance is always positive.

\section{Properties of the variance}
From its definition, we can derive properties of the variance function:

\begin{itemize}
  \item $\Var(X) \geq 0$ for all $X$
  \item Iff $X$ is constant, that is $\mathbb{P}(X = c) = 1$ for some $c$, then $\Var(X) = 0$
  \item $\Var(aX + b) = a^2\Var(X)$
  \item $\Var\left(\sum_{i = 1}^n a_iX_i\right) = \sum_{i = 1}^n a_i^2\Var(X_i)$ iff all the variables $X_i$ are independent
\end{itemize}

\subsection{Jensen's inequality}
From the properties of the variance we can derive the following property: if $g$ is a given \textit{convex} function, then $\E(g(X)) \geq g(\E(X))$. This comes from the fact that $\Var(X) = \E(X^2) - \E(X)^2 \geq 0$, defining $g(x) = x^2$. This holds for all convex functions, not just $g: x \mapsto x^2$.

\end{document}
