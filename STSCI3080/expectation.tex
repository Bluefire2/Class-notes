\documentclass[11pt]{amsart}

\usepackage{amssymb,amsmath}
\usepackage{bm}
\usepackage{mathtools}
\usepackage{framed}
\usepackage{esvect}
\usepackage{amsthm}
\usepackage{centernot}
\usepackage{ifxetex,ifluatex}

%%%%%%%%%%%%%%% FILL THIS IN FOR EACH ASSIGNMENT
\newcommand{\name}{Kyrylo Chernyshov}
\newcommand{\sectionnum}{203}
\newcommand{\norm}[1]{\left\lVert#1\right\rVert}
\newcommand*\diff{\mathop{}\!\mathrm{d}}
\newcommand*\Diff[1]{\mathop{}\!\mathrm{d^#1}}
\newcommand{\ex}{\subsubsection{Example}}
\newcommand*\mean[1]{\bar{#1}}

\DeclareMathOperator{\proj}{proj}
\DeclareMathOperator{\im}{im}
\DeclareMathOperator{\row}{row}
\DeclareMathOperator{\col}{col}
\DeclareMathOperator{\rank}{rank}
\DeclareMathOperator{\nullity}{nullity}
\DeclareMathOperator{\detm}{det}
\DeclarePairedDelimiter\ceil{\lceil}{\rceil}
\DeclarePairedDelimiter\floor{\lfloor}{\rfloor}
%%%%%%%%%%%%%%%%%%%%%%%%%%%%%%%%%%%%%%%%%%%%%%%%

\usepackage[margin=1in, letterpaper]{geometry}
\newcommand{\problem}[1]{\bigskip\noindent\textbf{Problem #1}}
\newcommand{\ppart}[1]{\bigskip\textbf{(#1)}}
\newcommand{\lemma}{\bigskip\textbf{Lemma}}
\newcommand{\E}{\mathrm{E}}
\newcommand{\Var}{\mathrm{Var}}
\newcommand{\Cov}{\mathrm{Cov}}

\begin{document}
\title{The expectation of a random variable}
\maketitle

\section{The expectation function}
The expectation function $\E$ gives the mean value of a random variable. It follows the following rules:

\begin{itemize}
  \item $\E(aX + b) = a\E(X) + b$
  \item $\E(X + Y) = \E(X) + \E(Y)$
  \item $\E(XY) = \E(X)\E(Y)$ iff $X$ and $Y$ are independent
\end{itemize}

\section{The expectation of a discrete random variable}
Suppose we have a variable $X$ such that $X: S \rightarrow D$ for some sample space $S$. As a result of the application of the formula for the expectation, we have

\begin{align*}
  \E(X) = \sum_D x f(x)
\end{align*}

\ex
Suppose we define $X$ such that $\mathbb{P}(X = x) = \frac{1}{x(x + 1)}$ for $x \in \mathbb{Z}^+$. This is a valid probability distribution, since $\sum_{i = 1}^{\infty} \mathbb{P}(X = i) = 1$. Using the formula above, we have

\begin{align*}
  \E(X) = \sum_{i = 1}^{\infty} \frac{i}{i(i + 1)} = \sum_{i = 1}^{\infty} \frac{1}{i + 1} = \infty
\end{align*}

\ex
To calculate the expected value of a binomially distributed variable, represent it as follows. Suppose there are $n$ Bernoulli distributed variables $X_i$ such that $\mathbb{P}(X = 1) = p$ and $\mathbb{P}(X = 0) = 1 - p$. Then, $X = \sum_{i = 0}^n X_i$ is binomially distributed with $X \thicksim B(n, p)$, since it is the sum of successes from $n$ trials with probability of success $p$. Using the rules listed above, we have

\begin{align*}
  \E(X) = \E \left(\sum_{i = 0}^n X_i\right) = \sum_{i = 0}^n \E(X_i) = np
\end{align*}

\section{The expectation of a continuous random variable}

\begin{align*}
  \E(X) = \int_{-\infty}^{\infty} x f(x) \diff x
\end{align*}

if at least one of $\int_{-\infty}^{0} x f(x) \diff x$ and $\int_{0}^{\infty} x f(x) \diff x$ is finite.

\ex
Suppose we have a device that will fail after a certain period of time, and that the time until the device fails is modelled by $X \thicksim f(x)$ where

\begin{align*}
  f(x) =
  \begin{cases}
    2x & 0 < x < 1 \\
    0 & \text{otherwise}
  \end{cases}
\end{align*}

In this case, $\E(X) = \int_0^1 (x \cdot 2x) \diff x = 2 \int_0^1 x^2 \diff x = 2 \frac{x^3}{3} \Big|_0^1 = \frac{2}{3}$.

\ex
An example of a pathological case is the Cauchy distribution: a special kind of continuous distribution whose p.d.f. is defined as $f(x) = \frac{1}{\pi(1 + x^2)}$. Suppose we have $X \thicksim f(x)$. We know that

\begin{align*}
  \int_{-\infty}^{0} \frac{x}{\pi(1 + x^2)} &= -\infty \\
  \int_0^{\infty} \frac{x}{\pi(1 + x^2)} &= \infty
\end{align*}

Therefore, $\E(X) = -\infty + \infty$, which is undefined, so this distribution \textit{does not have a mean}!.

\section{The expectation of a function of random variables}

\subsection{Theorem}
Let $X$ be a random variable, and let $Y = r(X)$ for $r: A \subseteq \mathbb{R} \rightarrow B \subseteq \mathbb{R}$. We have

\begin{align*}
  \E(Y) = \sum_B r(x) f(x)
\end{align*}

if $X$ is discrete, and

\begin{align*}
  \E(Y) = \int_{-\infty}^{\infty} r(x) f(x) \diff x
\end{align*}

if $X$ is continuous. These results follow directly from the \textit{Law of the Unconscious Statistician}.

\ex
Let $X$ represent the rate of failure of a certain machine per year, and let $Y = \frac{1}{X}$ be the time taken to fail. Suppose $X \thicksim f(x)$ where

\begin{align*}
  f(x) =
  \begin{cases}
    3x^2 & 0 < x < 1 \\
    0 & \text{otherwise}
  \end{cases}
\end{align*}

In order to find the expected amount of time until the first failure, we calculate

\begin{align*}
  \E(Y) = \int_0^1 \frac{1}{x} \cdot 3x^2 \diff x = 3 \int_0^1 x \diff x = 3 \frac{x^2}{2} \Bigg|_0^1 = \frac{3}{2}
\end{align*}

\subsection{Theorem}
Let $(X_1, X_2) \thicksim f(x_1, x_2)$, and let $Y = r(X_1, X_2)$ for $r: A \subseteq \mathbb{R}^2 \rightarrow B \subseteq \mathbb{R}^2$. Then, for continuous $X$

\begin{align*}
  \E(Y) = \int_{-\infty}^{\infty} \int_{-\infty}^{\infty} r(x_1, x_2) f(x_1, x_2) \diff x_1 \diff x_2
\end{align*}

\ex
Let $(X, Y) \thicksim f(x, y)$ where, for some $S \subseteq \mathbb{R}^2$

\begin{align*}
  f(x, y) =
  \begin{cases}
    1 & (x, y) \in S \\
    0 & \text{otherwise}
  \end{cases}
\end{align*}

Let $Z = X^2 + Y^2$, i.e. the squared distance of $(X, Y)$ from the origin. Suppose $S = \left\{(x, y) | 0 < x < 1, 0 < y < 1\right\}$. To calculate the mean squared distance of a point within $S$ from the origin, we use

\begin{align*}
  \E(Z) = \int_0^1 \int_0^1 (x^2 + y^2) f(x, y) \diff x \diff y = \int_0^1 \int_0^1 (x^2 + y^2) \diff x \diff y = \frac{2}{3}
\end{align*}

\end{document}
