\documentclass[11pt]{amsart}

\usepackage{amssymb,amsmath}
\usepackage{bm}
\usepackage{mathtools}
\usepackage{framed}
\usepackage{esvect}
\usepackage{amsthm}
\usepackage{centernot}
\usepackage{ifxetex,ifluatex}

%%%%%%%%%%%%%%% FILL THIS IN FOR EACH ASSIGNMENT
\newcommand{\name}{Kyrylo Chernyshov}
\newcommand{\sectionnum}{203}
\newcommand{\norm}[1]{\left\lVert#1\right\rVert}
\newcommand*\diff{\mathop{}\!\mathrm{d}}
\newcommand*\Diff[1]{\mathop{}\!\mathrm{d^#1}}
\newcommand{\ex}{\subsubsection{Example}}
\newcommand*\mean[1]{\bar{#1}}

\DeclareMathOperator{\proj}{proj}
\DeclareMathOperator{\im}{im}
\DeclareMathOperator{\row}{row}
\DeclareMathOperator{\col}{col}
\DeclareMathOperator{\rank}{rank}
\DeclareMathOperator{\nullity}{nullity}
\DeclareMathOperator{\detm}{det}
\DeclarePairedDelimiter\ceil{\lceil}{\rceil}
\DeclarePairedDelimiter\floor{\lfloor}{\rfloor}
%%%%%%%%%%%%%%%%%%%%%%%%%%%%%%%%%%%%%%%%%%%%%%%%

\usepackage[margin=1in, letterpaper]{geometry}
\newcommand{\problem}[1]{\bigskip\noindent\textbf{Problem #1}}
\newcommand{\ppart}[1]{\bigskip\textbf{(#1)}}
\newcommand{\lemma}{\bigskip\textbf{Lemma}}
\newcommand{\E}{\mathrm{E}}
\newcommand{\Var}{\mathrm{Var}}
\newcommand{\Cov}{\mathrm{Cov}}

\begin{document}
\title{STSCI 3080 HW \#4}
\author{\name}
\maketitle

\problem{3.8.4}
The c.d.f. of $X$ is $F(x) = \int_0^x f(t) \diff t = \int_0^x \frac{t}{2} \diff t = \frac{t^2}{4}\Big|_0^x = \frac{x^2}{2}$, for $0 \leq x \leq 2$. By definition, $F(x) = \mathbb{P}(X \leq x)$. Similarly, the c.d.f. of $Y$ is $G(y) = \mathbb{P}(Y \leq y) = \mathbb{P}(4 - X^3 \leq y) = \mathbb{P}(4 - y \leq X^3) = \mathbb{P}(\sqrt[3]{4 - y} \leq X) = 1 - \mathbb{P}(X \leq \sqrt[3]{4 - y}) = 1 - F(\sqrt[3]{4 - y}) = 1 - \frac{(4 - y)^{\frac{2}{3}}}{2}$, for $-4 \leq y \leq 0$. The p.d.f. of $Y$ is, therefore,

\begin{align*}
  g(y) &= \frac{\diff}{\diff y} G(y) \\
  &= \frac{\diff}{\diff y} \left(1 - \frac{(4 - y)^{\frac{2}{3}}}{2}\right) \\
  &= \frac{1}{3\sqrt[3]{4 - y}}
\end{align*}

\problem{3.8.6}
We have $Y = r(X)$ where $r(x) = 3x + 2$. Let $s(y) = r^{-1}(y) = \frac{y - 2}{3}$. Then, the p.d.f. of $y$ is
\begin{align*}
  g(y) &= f(s(y)) \cdot \frac{\diff s}{\diff y} \\
  &= \left(\frac{y - 2}{3}\right)^2 \frac{1}{2} \frac{1}{3} \\
  &= \frac{(y - 2)^2}{54}
\end{align*}

\problem{3.9.8}
For each $X_i$, its c.d.f. is $F(x) = \mathbb{P}(X \leq x) = x$, as it is a uniform distribution on the interval $[0, 1]$. By definition, the c.d.f. of $Y$ is $G(y) = \mathbb{P}(Y \leq y) = \mathbb{P}(X_1 \leq y, X_2 \leq y \cdots X_n \leq y) = \prod_{i = 1}^n \mathbb{P}(X_i \leq y) = \prod_{i = 1}^n F(y) = F(y)^n = y^n$. $\mathbb{P}(Y_n \geq 0.99) = 1 - \mathbb{P}(Y_n \leq 0.99) = 1 - G(0.99) = 1 - 0.99^n$. We need to find the smallest $n$ such that $\mathbb{P}(Y_n \geq 0.99) \geq 0.95$, so:

\begin{align*}
  \mathbb{P}(Y_n \geq 0.99) &\geq 0.95 \\
  1 - 0.99^n &\geq 0.95 \\
  0.99^n &\leq 0.05 \\
  n &\geq \log_{0.99}{0.05} \\
  &\geq 298.073
\end{align*}

So the smallest integral value is $n = 299$.

\problem{4.1.9}
Let $X$ be the distance from a certain end of the stick to the point where it is broken; $X$ is therefore uniformly distributed on $[0, 1]$. Define $Y$ as the length of the longer piece of the stick: $Y = \text{max}(X, 1 - X)$. Since $Y = X$ if $X \geq 1 - X$ and $Y = 1 - x$ if $X < 1 - X$, this means $Y$ takes on uniformly distributed values in the interval $[0.5, 1]$, and so its expected value is $\frac{0.5 + 1}{2} = 0.75$.

\problem{4.2.3}
Because they are uniformly distributed on $[0, 1]$, $\E(X_1) = \E(X_2) = \E(X_3) = 0.5$. Moreover, $\E(X_1^2) = \E(X_2^2) = \E(X_3^2) = \int_0^1 x^2 \diff x = \frac{x^3}{3} \Big|_0^1 = \frac{1}{3}$. Therefore:

\begin{align*}
  \E[(X_1 - 2X_2 + X_3)^2] &= \E[X_1^2 - 2X_1X_2 + X_1X_3 - 2X_2X_1 + 4X_2^2 - 2X_2X_3 + X_3X_1 - 2X_2X_3 + X_3^2] \\
  &= \E[X_1^2 + 4X_2^2 + X_3^2 - 4X_1X_2 - 4X_2X_3 + 2X_3X_1] \\
  &= \E(X_1^2) + 4\E(X_2^2) + \E(X_3^2) - 4\E(X_1X_2) - 4\E(X_2X_3) + 2\E(X_3X_1) \\
  &= \E(X_1^2) + 4\E(X_2^2) + \E(X_3^2) - 4\E(X_1)\E(X_2) - 4\E(X_2)\E(X_3) + 2\E(X_3)\E(X_1) \\
  &= \frac{1}{3} + 4\frac{1}{3} + \frac{1}{3} - 4\frac{1}{2}\frac{1}{2} - 4\frac{1}{2}\frac{1}{2} + 2\frac{1}{2}\frac{1}{2} \\
  &= \frac{1}{2}
\end{align*}

\problem{4.3.3}
Given the p.d.f. $f(x) = \frac{1}{b - a}$, the variance can be calculated as

\begin{align*}
  \Var(X) &= \int_a^b \frac{x^2}{b - a} \diff x - \left(\int_a^b \frac{x}{b - a} \diff x\right)^2 \\
  &= \frac{x^3}{3(b - a)} \Big|_a^b - \left(\frac{x^2}{2(b - a)} \Big|_a^b\right)^2 \\
  &= \frac{b^3 - a^3}{3(b - a)} - \left(\frac{b^2 - a^2}{2(b - a)}\right)^2 \\
  &= \frac{b^3 - a^3}{3(b - a)} - \frac{(a + b)^2}{4} \\
  &= \frac{(b -a)^2}{12}
\end{align*}

\problem{4.3.7}
$\Var(X - Y) = \Var(X) + (-1)^2\Var(Y) = \Var(X) + \Var(Y) = 6$. $\Var(2X − 3Y + 1) = 2^2\Var(X) + (-3)^2\Var(Y) = 4\Var(X) + 9\Var(Y) = 12 + 27 = 39$.

\problem{4.5.6}
By Theorem 4.5.2 in the book, the value of $d$ that minimises $\E[(X - d)^2]$ is the mean of $X$, that is, $\int_0^1 2x^2 \diff x = \frac{2x^3}{3} \Big|_0^1 = \frac{2}{3}$. By Theorem 4.5.3 in the book, the value of $d$ that minimises $\E[|X - d|]$ is the median of $X$, that is, the value $d$ such that $\mathbb{P}(X \leq d) = F(d) = \frac{1}{2}$. $\mathbb{P}(X \leq d) = \int_0^d 2x \diff x = x^2 \Big|_0^d = d^2$. The equality is, therefore, $d^2 = \frac{1}{2}$, so $d = \frac{\sqrt{2}}{2}$.

\end{document}
