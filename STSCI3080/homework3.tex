\documentclass[11pt]{amsart}

\usepackage{amssymb,amsmath,bm}
\usepackage{mathtools}
\usepackage{framed}
\usepackage{esvect}
\usepackage{amsthm}
\usepackage{centernot}
\usepackage{ifxetex,ifluatex}
\usepackage{fixltx2e} % provides \textsubscript
\usepackage{ulem}

%%%%%%%%%%%%%%% FILL THIS IN FOR EACH ASSIGNMENT
\newcommand{\name}{Kirill Chernyshov}
\newcommand{\hwnumber}{3}
\newcommand{\norm}[1]{\left\lVert#1\right\rVert}
\newcommand*\diff{\mathop{}\!\mathrm{d}}
\newcommand*\Diff[1]{\mathop{}\!\mathrm{d^#1}}

\DeclareMathOperator{\proj}{proj}
\DeclareMathOperator{\im}{im}
\DeclareMathOperator{\row}{row}
\DeclareMathOperator{\col}{col}
\DeclareMathOperator{\rank}{rank}
\DeclareMathOperator{\nullity}{nullity}
\DeclareMathOperator{\detm}{det}
\DeclarePairedDelimiter{\ceil}{\lceil}{\rceil}
%%%%%%%%%%%%%%%%%%%%%%%%%%%%%%%%%%%%%%%%%%%%%%%%

\usepackage[margin=1in, letterpaper]{geometry}
\newcommand{\problem}[1]{\bigskip\noindent\textbf{Problem #1}}
\newcommand{\ppart}[1]{\bigskip\textbf{(#1)}}
\begin{document}
\title{STSCI 3080 HW \#\hwnumber}
\author{\name}
\maketitle

%%%%%%%%%%%%%%%%%% BEGIN TYPING SOLUTIONS HERE

\problem{3.4.2}
This is simply a matter of adding up entries in the table.

\ppart{a}
Using the law of total probability, we need to add all the entries in the table for which $X = 2$, that is, $\sum_{i = 0}^4 \mathbb{P}(X = 2 | Y = i)$. This is equal to $0.05 + 0.06 + 0.09 + 0.04 + 0.03 = 0.27$.

\ppart{b}
$\mathbb{P}(Y \geq 2) = \mathbb{P}(Y = 2) + \mathbb{P}(Y = 3) + \mathbb{P}(Y = 4)$. Calculating these values as before, this is equal to $(0.06 + 0.12 + 0.09 + 0.03) + (0.01 + 0.05 + 0.04 + 0.03) + (0.01 + 0.02 + 0.03 + 0.04) = 0.53$.

\ppart{c}
Once again, summing the entries for which this is true, we get $(0.08 + 0.06 + 0.05) + (0.07 + 0.10 + 0.06) + (0.06 + 0.12 + 0.09) = 0.69$.

\ppart{d}
For this we need to sum the entries on the main diagonal: $0.08 + 0.10 + 0.09 + 0.03 = 0.3$.

\ppart{e}
For this we need to sum the entries strictly below the main diagonal: $(0.06 + 0.05 + 0.02) + (0.06 + 0.03) + (0.03) = 0.25$.

\problem{3.4.4}

\ppart{a}
From the laws of probability we know that $\int_0^1 \int_0^2 f(x, y) \diff x \diff y = 1$. $\int_0^1 \int_0^2 f(x, y) \diff x \diff y = c\int_0^1 \int_0^2 y^2 \diff x \diff y = c\int_0^1  xy^2 \Big|_0^2 \diff y = c\int_0^1  2y^2 \diff y = c \frac{2y^3}{3} \Big|_0^1 = \frac{2}{3}c$. Since $\frac{2}{3}c = 1$, we have $c = \frac{3}{2}$.

\ppart{b}
$\mathbb{P}(X + Y > 2) = \mathbb{P}(X > 2 - Y)$. This can be calculated by adjusting the integration limits: $\int_0^1 \int_{2 - y}^2 \frac{2y^2}{3} \diff x \diff y$. Since $\int_{2 - y}^2 \frac{2y^2}{3} \diff x = \frac{2xy^2}{3} \Big|_{2 - y}^2 = \frac{4y^2}{3} - \frac{2(2 - y)y^2}{3} = \frac{4y^2 - 4y^2 + 2y^3}{3} = \frac{2y^3}{3}$, then we have $\int_0^1 \int_{2 - y}^2 \frac{2y^2}{3} \diff x \diff y = \int_0^1 \frac{2y^3}{3} \diff y = \frac{y^4}{6} \Big|_0^1 = \frac{1}{6}$.

\ppart{c}
Using the law of total probability, this is equivalent to calculating the total probability that $\mathbb{P}\left(Y < \frac{1}{2}\right)$ for all values of $X$. This means calculating $\int_0^{\frac{1}{2}} \int_0^2 \frac{2y^2}{3} \diff x \diff y$. We know from before that $\int_0^2 \frac{2y^2}{3} \diff x = \frac{4y^2}{3}$, so $\int_0^{\frac{1}{2}} \int_0^2 \frac{2y^2}{3} \diff x \diff y = \int_0^{\frac{1}{2}} \frac{4y^2}{3} \diff y = \frac{4y^3}{9} \Big|_0^{\frac{1}{2}} = \frac{1}{18}$.

\ppart{d}
Similarly, for this we need to evaluate $\int_0^1 \int_0^1 \frac{2y^2}{3} \diff x \diff y$. Since $\int_0^1 \frac{2y^2}{3} \diff x = \frac{2y^2}{3}$, we have $\int_0^1 \int_0^1 \frac{2y^2}{3} \diff x \diff y = \int_0^1 \frac{2y^2}{3} \diff y = \frac{2}{9}$.

\ppart{e}


\problem{3.5.2}

\ppart{a}
The marginal p.f. of $X$ is $f_1(x) = \frac{1}{30}\sum_{i = 0}^3 f(x, i) = \frac{1}{30}(x + 0 + x + 1 + x + 2 + x + 3) = \frac{4x + 6}{30} = \frac{2x + 3}{15}$. The marginal p.f. of $Y$ is $f_2(y) = \frac{1}{30}\sum_{i = 0}^2 f(i, y) = \frac{1}{30}(0 + y + 1 + y + 2 + y) = \frac{3y + 3}{30} = \frac{y + 1}{10}$.

\ppart{b}
$X$ and $Y$ are independent if and only if $f_1(x)f_2(y) = f(x, y)$. $f_1(x)f_2(y) = \frac{2x + 3}{15}\frac{y + 1}{10} = \frac{2xy + 2x + 3y + 3}{150} \neq f(x, y)$. Therefore, $X$ and $Y$ are not independent.

\problem{3.5.4}

\ppart{a}
The marginal p.d.f. of $X$ is $f_1(x) = \frac{15}{4}\int_0^{1 - x^2} x^2 \diff y = \frac{15}{4}x^2(1 - x^2)$. Since $0 \leq y \leq 1 - x^2$, we have $0 \leq x^2 \leq 1 - y \implies x \leq \sqrt{1 - y}$ The marginal p.d.f. of $Y$ is $\frac{15}{4}\int_0^{\sqrt{1 - y}} x^2 \diff x = \frac{15}{4}\frac{(\sqrt{1 - y})^3}{3}$.

\ppart{b}
Since the set $\left\{ (x, y) : f(x, y) > 0 \right\}$ has a curved boundary (since $y$ depends on $1 - x^2$), $X$ and $Y$ cannot be independent.

\problem{3.5.6}

\ppart{a}
Since $X$ and $Y$ are independent, $f(x, y) = g(x)g(y) = \frac{9}{64}x^2y^2$.

\ppart{b}

\ppart{c}
$\mathbb{P}(X > Y) = \int_0^2 \int_0^x f(x, y) \diff y \diff x = \frac{9}{64}\int_0^2 \int_0^x x^2y^2 \diff y \diff x = \frac{9}{64}\int_0^2 \frac{x^5}{3} \diff x = \frac{9}{64}\frac{32}{9} = \frac{1}{2}$.

\ppart{d}
$\mathbb{P}(X + Y \leq 1) = \mathbb{P}(X \leq 1 - Y) = \int_0^2 \int_0^{1 - y} f(x, y) \diff x \diff y = \frac{9}{64}\int_0^2 \int_0^{1 - y} x^2y^2 \diff x \diff y = \frac{9}{64}\int_0^2 \frac{(1 - y)^3y^2}{3} \diff y$. This evaluates to

\problem{3.5.13}
$x^2 + y^2 \leq 1 \implies y^2 \leq 1 - x^2 \implies -\sqrt{1 - x^2} \leq y \leq \sqrt{1 - x^2}$. Therefore, $f_1(x) = k\int_{-\sqrt{1 - x^2}}^{\sqrt{1 - x^2}} x^2y^2 \diff y = k \frac{x^2y^3}{3} \Big|_{-\sqrt{1 - x^2}}^{\sqrt{1 - x^2}} = 2k \frac{x^2y^3}{3} \Big|_0^{\sqrt{1 - x^2}} = \frac{2kx^2\sqrt{1 - x^2}^3}{3}$. The range for $x$ is $-1 \leq x \leq 1$, since $x^2 + y^2 \leq 1 \implies -\sqrt{1 - y^2} \leq x \leq \sqrt{1 - y^2}$, and the largest possible value of $\sqrt{1 - y^2}$ is $1$ (when $y = 0$). Since $x$ and $y$ are interchangeable in both the range $x^2 + y^2 = 1$ and the function $f(x, y) = kx^2y^2$, this means they have the same marginal distributions.

\problem{3.6.4}

\ppart{a}
By definition, the conditional p.d.f. is $g_1(x|y) = \frac{f(x, y)}{f_2{y}}$. $f_2(y) = \int_0^1 f(x, y) \diff x = c\int_0^1 (x + y^2) \diff x = c\left( \frac{x^2}{2} + xy^2 \right) \Big|_0^1 = c\left(y^2 + \frac{1}{2}\right)$. \sout{To calculate $c$ we can use the fact that $c\int_0^1 \int_0^1 x + y^2 \diff x \diff y = 1 = c\frac{5}{6}$, so $c = \frac{6}{5}$. Therefore, $f_2(y) = \frac{6}{5}\left(y^2 + \frac{1}{2}\right)$}.

Therefore, $g_1(x|y) = \frac{f(x, y)}{f_2(y)} = \frac{c(x + y^2)}{c\left(y^2 + \frac{1}{2}\right)} = \frac{x + y^2}{y^2 + \frac{1}{2}}$

\ppart{b}
Using the function from before, $\mathbb{P}\left(X > \frac{1}{2} | Y = \frac{3}{4}\right) = \int_{\frac{1}{2}}^1 \frac{x + \frac{9}{16}}{\frac{9}{16} + \frac{1}{2}} \diff x = 0.602$.

\problem{3.6.8}

\ppart{a}
$\mathbb{P}(X > 0.8) = \frac{2}{5}\int_0^1 \int_0^{0.8} (2x + 3y) \diff x \diff y = \frac{2}{5}\int_0^1 2.4y + 0.64 \diff y = 1.84\frac{2}{5} = 0.736$.

\ppart{b}
$f_2(y) = \frac{2}{5}\int_0^1 (2x + 3y) \diff x = \frac{2}{5}(3y + 1)$. Therefore, $g_1(x | y) = \frac{f(x, y)}{f_2(y)} = \frac{2x + 3y}{3y + 1}$. By definition, $\mathbb{P}(X > 0.8 | Y = 0.3) = \int_0^{0.8} g_2(x | 0.3) \diff x = \int_0^{0.8} \frac{2x + 0.9}{0.9 + 1} \diff x = 0.716$.

\ppart{c}
$f_1(x) = \frac{2}{5}\int_0^1 (2x + 3y) \diff y = \frac{2}{5}\left(2x + \frac{3}{2}\right)$. Therefore, $g_2(y | x) = \frac{f(x, y)}{f_1(x)} = \frac{2x + 3y}{2x + \frac{3}{2}}$. By definition, $\mathbb{P}(Y > 0.8 | X = 0.3) = \int_0^{0.8} g_2(y | 0.3) \diff y = \int_0^{0.8} \frac{0.6 + 3y}{0.6 + \frac{3}{2}} \diff y = 0.686$.

\end{document}
