\documentclass[11pt]{amsart}

\usepackage{amssymb,amsmath}
\usepackage{bm}
\usepackage{mathtools}
\usepackage{framed}
\usepackage{esvect}
\usepackage{amsthm}
\usepackage{centernot}
\usepackage{ifxetex,ifluatex}

%%%%%%%%%%%%%%% FILL THIS IN FOR EACH ASSIGNMENT
\newcommand{\name}{Kirill Chernyshov}
\newcommand{\sectionnum}{203}
\newcommand{\hwnumber}{7}
\newcommand{\norm}[1]{\left\lVert#1\right\rVert}
\newcommand*\diff{\mathop{}\!\mathrm{d}}
\newcommand*\Diff[1]{\mathop{}\!\mathrm{d^#1}}

\DeclareMathOperator{\proj}{proj}
\DeclareMathOperator{\im}{im}
\DeclareMathOperator{\row}{row}
\DeclareMathOperator{\col}{col}
\DeclareMathOperator{\rank}{rank}
\DeclareMathOperator{\nullity}{nullity}
\DeclareMathOperator{\detm}{det}
\DeclarePairedDelimiter{\ceil}{\lceil}{\rceil}
%%%%%%%%%%%%%%%%%%%%%%%%%%%%%%%%%%%%%%%%%%%%%%%%

\usepackage[margin=1in, letterpaper]{geometry}
\newcommand{\problem}[1]{\bigskip\noindent\textbf{Problem #1}}
\newcommand{\ppart}[1]{\bigskip\textbf{(#1)}}
\begin{document}
\title{Constrained extrema and Lagrange multipliers}
\author{\name}
\maketitle

%%%%%%%%%%%%%%%%%% BEGIN TYPING SOLUTIONS HERE

Let $f$ and $g$ be $C^1$ functions $E \subseteq \mathbb{R}^3 \rightarrow \mathbb{R}$. Let $S$ be the level set $g(x, y, z) = c$. Let $\bf{p} \in S$ be a vector such that $\nabla g(\bf{p}) \neq 0$ and $\bf{p}$ is a local extremum of $f$ on $S$. Then, there exists a scalar $\lambda$ such that $\nabla f(\bf{p}) = \lambda \nabla g(\bf{p})$.
\bigskip
\begin{proof}
  Assume that $\nabla g(\bf{p}) \neq 0$. This means there exists a co-ordinate in $\bf{p}$ that is nonzero. Without loss of generality we can assume that $z$ is that co-ordinate. By the implicit function theorem, in an open set around $\bf{p}$ we can solve for $z$ in terms of $x$ and $y$. That is, there exists $\phi: U \subseteq \mathbb{R}^2 \rightarrow \mathbb{R}$ such that $g(x, y, \phi(x, y)) = c$ and $\phi(p_1, p_2) = p_3$.

  Define $h(x, y) = f(x, y, \phi(x, y))$. $(p_1, p_2)$ is a local extremum of $h$, since $\bf{p}$ is an extremum of $f$. Therefore, $h_x(p_1, p_2) = 0$, which is equal to, using the chain rule, $\frac{\partial f}{\partial x}\bf{p} + \frac{\partial f}{\partial z} \cdot \frac{\partial \phi}{\partial x}(p_1, p_2)$. The same steps can be applied to $h_y$.

  We know that $g(x, y, \phi(x, y)) = c$, so $g_x(x, y, \phi(x, y)) = \frac{\partial g}{\partial x}(\bf{p}) + \frac{\partial g}{\partial z}(\bf{p}) \cdot \frac{\partial \phi}{\partial x}(p_1, p_2) = 0$, which in turn means that $\frac{\partial \phi}{\partial x}(p_1, p_2) = -\frac{\partial g}{\partial x}(\bf{p}) \cdot \left(\frac{\partial g}{\partial z}(\bf{p})\right)^{-1}$.

  Using the equation from before, we can plug in the value for $\frac{\partial \phi}{\partial x}(p_1, p_2)$, we have $\frac{\partial f}{\partial x}(\bf{p}) = \frac{\frac{\partial f}{\partial z}(\bf{p})}{\frac{\partial g}{\partial z}(\bf{p})} \cdot \frac{\partial g}{\partial x}(\bf{p})$. This gives $\lambda = \frac{\frac{\partial f}{\partial z}(\bf{p})}{\frac{\partial g}{\partial z}(\bf{p})}$.
\end{proof}
\bigskip
We can use this to find constrained extrema. For example, say we are given an ellipsoid in $\mathbb{R}^3$, and we want to find the box with the greatest volume that is inscribed in the ellipsoid. This is an example of constrained optimisation.

Let $3x^2 + 5y^2 + 7z^2 = 1$ be the level set that describes the ellipsoid. If one vertex of the box is $(a, b, c)$ then all the other vertices will be of the form $(\pm a, \pm b, \pm c)$. Therefore, the volume of this box will be $(2a)(2b)(2c) = 8abc$. Therefore we have $f(x, y, z) = 8xyz$ as the function that describes the volume, and $g(x, y, z) = 3x^2 + 5y^2 + 7z^2 = 1$ as the level set constraint.

$\nabla f = (8yz, 8xz, 8xy)$ and $\nabla g = (6x, 10y, 14z)$. We need to find $\bf{p}$ such that $\nabla f(\bf{p}) = \lambda_p \nabla f(\bf{p})$:

$$
\begin{array}{lcl}
  3x^2 + 5y^2 + 7z^2 = 1 \\
  8yz = 6 \lambda_p x \\
  8xz = 10 \lambda_p y \\
  8xy = 14 \lambda_p z
\end{array}
$$

These four equations are in four variables, so they can be solved for $x$, $y$, $z$ and $\lambda_p$ (we can assume $x, y, z > 0$, and therefore $\lambda_p \neq 0$).

Dividing equation 2 by equation 3, we get $\frac{y}{x} = \frac{3x}{5y} \implies 5y^2 = 3x^2 \implies y = \sqrt{\frac{3x}{5}}$. Dividing equation 3 by equation 4, we get $\frac{z}{y} = \frac{5y}{7z} \implies 5y^2 = 7z^2 \implies z = \sqrt{\frac{5y}{7}} = \sqrt{\frac{3x}{7}}$. We can put this into the first equation to get $3x^2 + 3x^2 + 3x^2 = 1 \implies x = \frac{1}{3}, y = \frac{1}{\sqrt{15}}, z = \frac{1}{\sqrt{21}}$, and the volume is $8xyz = \frac{8}{9\sqrt{35}}$.

\end{document}
