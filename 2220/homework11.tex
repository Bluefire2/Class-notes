\documentclass[12pt]{article}
\usepackage{fancyhdr}     % Enhanced control over headers and footers
\usepackage[T1]{fontenc}  % Font encoding
\usepackage{mathptmx}     % Choose Times font
\usepackage{microtype}    % Improves line breaks
\usepackage{setspace}     % Makes the document look like horse manure
\usepackage{lipsum}       % For dummy text
\usepackage{etoolbox}

\AtBeginEnvironment{quote}{\singlespacing\small}

\newcommand{\name}{Kirill Chernyshov}

\pagestyle{fancy} % Default page style
\lhead{\name}
\chead{}
\rhead{\thepage}
\lfoot{}
\cfoot{}
\rfoot{}
\renewcommand{\headrulewidth}{1pt}
\renewcommand{\footrulewidth}{1pt}

\thispagestyle{empty} %First page style

\setlength\headheight{15pt} %Slight increase to header size

\begin{document}
\begin{center}
\begin{tabular}{c}
\textbf{\name} \\
\textbf{\today}
\end{tabular}
\end{center}
\doublespacing

\usepackage[normalem]{ulem}
\begin{document}
\title{Math 2220 HW \#11}
\author{\name}
\maketitle

\problem{1}

If two integals are calculated over sets $C_1$ and $C_2$ respectively, such that their intersection is either the empty set or a subset of their boundary points, then their sum is the integral over $C_1 \cup C_2$. In this case, $C_1 \cup C_2 = C$, and all points on a line are boundary points.

\problem{2}

\ppart{a}
Let $r(t) = (R \cos t, R \sin t)$. Then, $r'(t) = (-R \sin t, R \cos t)$. By definition:

\begin{align*}
  \int_C \vv{f} \cdot \vv{t} \diff S &= \int_0^{2\pi} \vv{f}(r(t)) \cdot r'(t) \diff t \\
  &= \int_0^{2\pi} \vv{f}((R \cos t, R \sin t)) \cdot (-R \sin t, R \cos t) \diff t \\
  &= \int_0^{2\pi} \left(\frac{-\sin t}{R}, \frac{\cos t}{R}\right) \cdot (-R \sin t, R \cos t) \diff t \\
  &= \int_0^{2\pi} (\sin^2 t + \cos^2 t) \diff t = \int_0^{2\pi} \diff t = 2\pi
\end{align*}

\ppart{b}
\sout{$\vv{f}$ voted for Obama}
$C$ starts and ends at the same point, since $r(0) = r(2\pi) = (R, 0)$. In a conservative vector field, any line integral over a curve that loops back to the same point would be $0$, as a line integral in such a field is path-independent. However, the integral over $C$ is $2\pi \neq 0$. Therefore, $\vv{f}$ is not conservative.

\problem{3}

\ppart{a}
Since the set $C_1$ is symmetric about both $x$ and $y$, $\int_{C_1} x^2 \diff S = \int_{C_1} y^2 \diff S$. Therefore, $\int_C (x^2 + y^2) \diff S = \int_{C_1} x^2 \diff S + \int_{C_1} y^2 \diff S = 2\int_{C_1} x^2 \diff S$.

\ppart{b}
For the same reason, this works for $C_2$.

\ppart{c}
On the unit circle, every point has the property that $x^2 + y^2 = 1$. Therefore, $\int_{C_1} (x^2 + y^2)^{10} \diff S = \int_C \diff S = \text{length}(C_1)$.

\ppart{d}
Once again, $\text{length}(C_2) = \int_{C_2} \diff S$. On the unit square, the closest points to the origin are the points of intersection with the $x$- and $y$-axes, where $x = \pm 1, y = 0$ or $x = 0, y = \pm 1$. Their squared norm is their distance to the origin squared, i.e. $1$. This means that $(x^2 + y^2)^{10} \geq x^2 + y^2 \geq 1 \;\forall\; x, y \in C_2$, so $\int_{C_2} (x^2 + y^2)^{10} \diff S > \int_{C_2} \diff S$.

\problem{4}
The function that parametrises the set of points $C$ where $x^2 + y^2 = r^2$ is $s(t) = (r\cos t, r\sin t)$, for $0 \leq r \leq 2\pi$. and $s'(t) = (-r\sin t, r\cos t)$. Therefore:

\begin{align*}
  \int_C (-3y, 2x) \cdot \vv{t} \diff S &= \int_0^{2\pi} (-3r\sin t, 2r\cos t) \cdot (-r\sin t, r\cos t) \diff t \\
  &= \int_0^{2\pi} (3r^2 \sin^2 t + 2r^2 \cos^2 t) \diff t = \int_0^{2\pi} (r^2 \sin^2 t + 2r^2 (\sin^2 t + \cos^2 t) \diff t = \int_0^{2\pi} (r^2 \sin^2 t + 2r^2 \diff t \\
  &= r^2 \int_0^{2\pi} (\sin^2 t + 2) \diff t = r^2\left(\frac{5t}{2} - \frac{\sin(2t)}{4}\right)\Big|_0^{2\pi} = 5\pi r^2
\end{align*}

Since the second vector field is the first vector field multiplied by $-5$, its rotation is $\int_C -5(-3y, 2x) \cdot \vv{t} \diff S = -5\int_C (-3y, 2x) \cdot \vv{t} \diff S = -25\pi r^2$.

\problem{5}

Given this information, we have the following equalities:

\begin{align*}
  \frac{\partial}{\partial x} g = a \\
  \frac{\partial}{\partial y} g = b \\
  \frac{\partial}{\partial z} g = c \\
\end{align*}

Consider for example $a$ and $b$. Differentiating one with respect to $y$ and one with respect to $x$, we have $\frac{\partial}{\partial y} a = \frac{\partial}{\partial y}\frac{\partial}{\partial x} g$ and $\frac{\partial}{\partial x} b = \frac{\partial}{\partial x}\frac{\partial}{\partial y} g$. Since the order of differentiation does not matter, we have $\frac{\partial}{\partial y} a = \frac{\partial}{\partial x} b$. Similarly, we can derive the fact that $\frac{\partial}{\partial z} b = \frac{\partial}{\partial y} c$ and $\frac{\partial}{\partial z} a = \frac{\partial}{\partial x} c$.

\problem{6}

\ppart{a}
By definition, using the chain rule and then the fundamental theorem of calculus:

\begin{align*}
  \int_C \nabla g \cdot \vv{T} \diff S &= \int_0^1 \nabla g(\vv{\phi}(t)) \cdot \vv{\phi}'(t) \diff t \\
  &= \int_0^1 \nabla (g \circ \vv{\phi}) \diff t = (g \circ \vv{\phi})(1) - (g \circ \vv{\phi})(0)
\end{align*}

\ppart{b}
Suppose $C = \bigcup_{i = 0}^k C_i$ for some $k$, where the endpoints of the curves $C_1, C_2 \cdots C_k$ are $t = a, t = b$, $t = b, t = c$ et cetera. Then

\begin{align*}
  \int_C \nabla g \cdot \vv{T} \diff S &= \sum_{i = 0}^k \int_{C_i} \nabla g \cdot \vv{T} \diff S \\
  &= (g \circ \vv{\phi})(1) - (g \circ \vv{\phi})(a) + (g \circ \vv{\phi})(a) - (g \circ \vv{\phi})(b) + \cdots + (g \circ \vv{\phi})(n) - (g \circ \vv{\phi})(0) \\
  &= (g \circ \vv{\phi})(1) - (g \circ \vv{\phi})(0)
\end{align*}

\ppart{c}
Gravitational force $\vv{F}$ is defined in terms of gravitational potential $\vv{U}$ as $\vv{F} = -\nabla \vv{U}$. Therefore, gravity is a conservative vector field. This means that the total work done walking in a path that starts and ends at the same point, i.e. the line integral along the path, is $0$, assuming no friction etc. This is why it's impossible to go uphill both ways, as that would imply that there is a positive amount of work being done.

\ppart{d}
what even is this question

\end{document}
