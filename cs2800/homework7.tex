\documentclass[12pt]{article}
\usepackage{fancyhdr}     % Enhanced control over headers and footers
\usepackage[T1]{fontenc}  % Font encoding
\usepackage{mathptmx}     % Choose Times font
\usepackage{microtype}    % Improves line breaks
\usepackage{setspace}     % Makes the document look like horse manure
\usepackage{lipsum}       % For dummy text
\usepackage{etoolbox}

\AtBeginEnvironment{quote}{\singlespacing\small}

\newcommand{\name}{Kirill Chernyshov}

\pagestyle{fancy} % Default page style
\lhead{\name}
\chead{}
\rhead{\thepage}
\lfoot{}
\cfoot{}
\rfoot{}
\renewcommand{\headrulewidth}{1pt}
\renewcommand{\footrulewidth}{1pt}

\thispagestyle{empty} %First page style

\setlength\headheight{15pt} %Slight increase to header size

\begin{document}
\begin{center}
\begin{tabular}{c}
\textbf{\name} \\
\textbf{\today}
\end{tabular}
\end{center}
\doublespacing

\begin{document}
\title{CS 2800 HW \#6}
\author{\name}
\maketitle

\problem{1}

\ppart{a}
Suppose $q_1 \thicksim q_2$. Then, $\hat{\delta}(q_1, \epsilon) \in A \iff \hat{\delta}(q_2, \epsilon) \in A$. But $\hat{\delta}(q_1, \epsilon) = q_1$ and $\hat{\delta}(q_2, \epsilon) = q_2$, by the definition of the extended transition function. Therefore, $q_1 \in A \iff q_2 \in A$, i.e. if one state in an equivalence class is an accepting state then all other states in that class are also accepting states. This means that any equivalence class can contain only accepting states or only non-accepting states.

\ppart{b}
In this context, a function $f$ that is well defined satisfies the following property: if $a \thicksim b$ then $f(a) = f(b)$. To prove this by contradiction, assume that there exist $q_a$ and $q_b$ such that $q_a \thicksim q_b$ but $\delta_{min}(q_a, k)$ is not the same equivalence class as $\delta_{min}(q_b, k)$ for some $k \in \Sigma$.

\ppart{c}
Using the definition of $\delta_{min}$: suppose $\delta_{min}(q, a)$ is accepting for some $q$ and $a$. Then, we know that $\delta_{M}(q, a)$ is accepting, and vice versa if it's rejecting. By the theorem proven in part a, all states in an equivalence class are either accepting or rejecting. That means that if $\delta_{min}([q], a)$ is accepting then all states in $[\delta_{M}(q, a)]$ are accepting, and vice versa if it is rejecting. This means that $M_{min}$ accepts an input iff $M$ does, so their languages are the same.

\problem{2}

\ppart{a}
This proof would work for a DFA. It does not work for a NFA because $x$ is said to be recognisable by NFA $M$ if \textit{one} of the states that $x$ \textit{could} end up in is an accepting state. If $x$ can end up in either state $q_a$, an accepting state, or $q_r$, a rejecting state, inverting all the states doesn't stop $x$ from being accepted, since $q_r$ now becomes an accepting state, and $x$ can end up in $q_r$. Since $x$ is accepted by both the original NFA and the new inverted NFA, their languages cannot be complements: if they were, then any string accepted by one machine would be rejected by the other machine.

An example is simply as stated above: a machine $M$ that has three states, $q_a$, $q_r$ and the starting state $q_s$, that takes the alphabet $\left\{0, 1\right\}$, and upon $1$ transitions to either $q_a$ or $q_r$. 1 can end up in $q_a$, so it is in the language of $M$. If the states are inverted to form machine $N$, 1 is still accepted since it can end up in $q_r$, which is now accepting. Therefore 1 is also in the language of $N$, and so $L(M)$ and $L(N)$ cannot be complements.

\ppart{b}
Kleene's theorem implies that any NFA-recognisable language can be expressed as a regex. This is because if $L$ is NFA recognisable, there exists a NFA $M$ such that $L$ is the language of $M$. By Kleene's theorem, there exists a regex whose language is $L$. Again by Kleene's theorem, every regex can be expressed as a DFA that recognises the same language. From this DFA a ``complement'' DFA can be made by flipping all the states (accepted becomes rejected and vice versa); the flawed proof above proves that this can invert a DFA (not an NFA) and its language will be complementary to the previous DFA's language. However, a DFA is also a NFA. Since this NFA must exist, the complement of any NFA-recognisable language is also DFA-recognisable.

\ppart{c}
This DFA accepts the language denoted by the regex $aa + ab*$, since the only way to end up in $q_2$ is $ab*$ and the only way to end up in $q_4$ is $aa$ (the only two accepting states). We can construct a DFA $M$ with the same alphabet that accepts this regex. The states of $M$ are the starting state $q_s$, the hell state $q_h$ and the three accepting states $q_3$, $q_4$ and $q_5$. A string starts in the starting state (duh); upon $b$ it goes to hell and upon $a$ it proceeds to $q_3$. At $q_3$, upon $b$ the string proceeds to $q_4$ and upon $a$ to $q_5$. In $q_4$, upon $b$ the string remains in $q_4$ and upon $a$ goes to hell; in $q_5$ upon either $a$ or $b$ it goes to hell. In hell, upon either $a$ or $b$ the string remains in hell.

If we take this DFA and flip all of its states (the accepting states become rejecting and vice versa), it now accepts the complement of the above regex. Since it is also a NFA, it satisfies the properties required.

\ppart{d}
Once again we can create a DFA that recognises the regex and flip it to make its complement. This DFA has the same alphabet as above. It has starting state $q_s$, hell state $q_h$, and three accepting states, $q_3$, $q_4$ and $q_5$. $q_s$ is also accepting, as the regex accepts $\epsilon$. From $q_s$, $a$ transitions to $q_3$, $b$ goes to hell. From $q_3$, $a$ goes to $q_4$, $b$ goes to $q_5$. From $q_4$, $a$ remains in $q_4$ and $b$ transitions to hell, and from $q_5$ either character causes a transition to hell. In hell, any character will keep the string in hell.

To invert this DFA, make $q_s$, $q_3$, $q_4$ and $q_5$ rejecting, and $q_h$ accepting. Analysing this new DFA, it can be seen that it accepts the regex $b + aba + abb + aaa*b$. Since this DFA is complementary to the previous one, this regex must be the complement of $ab + a*$.
\end{document}
