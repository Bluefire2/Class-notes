\documentclass[11pt]{amsart}

\usepackage{amssymb,amsmath}
\usepackage{bm}
\usepackage{mathtools}
\usepackage{framed}
\usepackage{esvect}
\usepackage{amsthm}
\usepackage{centernot}
\usepackage{ifxetex,ifluatex}

%%%%%%%%%%%%%%% FILL THIS IN FOR EACH ASSIGNMENT
\newcommand{\name}{Kyrylo Chernyshov}
\newcommand{\sectionnum}{203}
\newcommand{\norm}[1]{\left\lVert#1\right\rVert}
\newcommand*\diff{\mathop{}\!\mathrm{d}}
\newcommand*\Diff[1]{\mathop{}\!\mathrm{d^#1}}
\newcommand{\ex}{\subsubsection{Example}}
\newcommand*\mean[1]{\bar{#1}}

\DeclareMathOperator{\proj}{proj}
\DeclareMathOperator{\im}{im}
\DeclareMathOperator{\row}{row}
\DeclareMathOperator{\col}{col}
\DeclareMathOperator{\rank}{rank}
\DeclareMathOperator{\nullity}{nullity}
\DeclareMathOperator{\detm}{det}
\DeclarePairedDelimiter\ceil{\lceil}{\rceil}
\DeclarePairedDelimiter\floor{\lfloor}{\rfloor}
%%%%%%%%%%%%%%%%%%%%%%%%%%%%%%%%%%%%%%%%%%%%%%%%

\usepackage[margin=1in, letterpaper]{geometry}
\newcommand{\problem}[1]{\bigskip\noindent\textbf{Problem #1}}
\newcommand{\ppart}[1]{\bigskip\textbf{(#1)}}
\newcommand{\lemma}{\bigskip\textbf{Lemma}}
\newcommand{\E}{\mathrm{E}}
\newcommand{\Var}{\mathrm{Var}}
\newcommand{\Cov}{\mathrm{Cov}}

\usepackage{proof}
\usepackage{turnstile}
\begin{document}
\title{CS 2800 HW \#10}
\author{\name}
\maketitle

Key: as. = assumption, abs. = reductio ad absurdum, e. m. = excluded middle.

\problem{1}

\ppart{a}
Suppose $\neg P$. $Q$ can either evaluate to $T$ or $F$:

\begin{center}
\begin{tabular}{lcccc}
$\neg P$ & $P$ & $Q$ & $P \land Q$ & $\neg(P \land Q)$ \\ \hline
T & F & T & F & T \\
T & F & F & F & T
\end{tabular}
\end{center}

Therefore, $\neg P \models \neg (P \land Q)$.

\ppart{b}

\infer[\lor\; elim]{\neg P \vdash \neg(P \land Q)}{
  \infer[e.\; m.]{\vdash (P \land Q) \lor \neg(P \land Q)}{}
  &
  \infer[abs.]{\neg P, P \land Q \vdash \neg(P \land Q)}{
    \infer[\land\; elim]{P \land Q \vdash P}{
      \infer[as.]{P \land Q \vdash P \land Q}{}
    }
    &
    \infer[as.]{\neg P \vdash \neg P}{}
  }
  &
  \infer[as.]{\neg(P \land Q) \vdash \neg(P \land Q)}{}
}

\problem{2}

Inductively assume $P(\phi)$ and $P(\psi)$. $I \models \phi \rightarrow \psi$ means that $\text{eval}(\phi \rightarrow \psi, I) = T$, that is, $I \models \neg\phi$ or $I \models \psi$. If $I \models \neg\phi$ then using our assumption, $I \vdash \neg\phi$, and so $I \vdash \phi \rightarrow \psi$. This can be seen from the following proof tree:

\begin{align*}
\infer[\rightarrow\; intro]{\neg \phi \vdash \phi \rightarrow \psi}{
  \infer[abs.]{\neg \phi, \phi \vdash \psi}{
    \infer[as.]{\phi \vdash \phi}{}
    &
    \infer[as.]{\neg \phi \vdash \neg phi}{}
  }
}
\end{align*}

If $I \models \psi$, then using our assumption, $I \vdash \psi$, and once again we have $I \vdash \phi \rightarrow \psi$:

\begin{align*}
\infer[\rightarrow\; intro]{\psi \vdash \phi \rightarrow \psi}{
  \infer[as.]{\phi, \psi \vdash \psi}{}
}
\end{align*}

Conversely, if $I \not\models \phi \rightarrow \psi$, then $I \models \phi \land \neg \psi$. Using the inductive assumption, we have $I \vdash \phi$ and $I \vdash \neg\psi$. Therefore:

\begin{align*}
\infer[\lor\; elim]{\phi, \neg \psi \vdash \neg(\phi \rightarrow \psi)}{
  \infer[e.\; m.]{\vdash (\phi \rightarrow \psi) \lor \neg(\phi \rightarrow \psi)}{}
  &
  \infer[abs.]{\phi, \neg \psi, \phi \rightarrow \psi \vdash \neg(\phi \rightarrow \psi)}{
    \infer[\rightarrow\; elim]{\phi, \phi \rightarrow \psi \vdash \psi}{
      \infer[as.]{\phi \vdash \phi}{}
      &
      \infer[as.]{\phi \rightarrow \psi \vdash \phi \rightarrow \psi}{}
    }
    &
    \infer[as.]{\neg \psi \vdash \neg \psi}{}
  }
  &
  \infer[as.]{\neg(\phi \rightarrow \psi) \vdash \neg(\phi \rightarrow \psi)}{}
}
\end{align*}

\end{document}
