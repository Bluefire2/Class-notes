\documentclass[11pt]{amsart}

\usepackage{amssymb,amsmath}
\usepackage{bm}
\usepackage{mathtools}
\usepackage{framed}
\usepackage{esvect}
\usepackage{amsthm}
\usepackage{centernot}
\usepackage{ifxetex,ifluatex}

%%%%%%%%%%%%%%% FILL THIS IN FOR EACH ASSIGNMENT
\newcommand{\name}{Kyrylo Chernyshov}
\newcommand{\sectionnum}{203}
\newcommand{\norm}[1]{\left\lVert#1\right\rVert}
\newcommand*\diff{\mathop{}\!\mathrm{d}}
\newcommand*\Diff[1]{\mathop{}\!\mathrm{d^#1}}
\newcommand{\ex}{\subsubsection{Example}}
\newcommand*\mean[1]{\bar{#1}}

\DeclareMathOperator{\proj}{proj}
\DeclareMathOperator{\im}{im}
\DeclareMathOperator{\row}{row}
\DeclareMathOperator{\col}{col}
\DeclareMathOperator{\rank}{rank}
\DeclareMathOperator{\nullity}{nullity}
\DeclareMathOperator{\detm}{det}
\DeclarePairedDelimiter\ceil{\lceil}{\rceil}
\DeclarePairedDelimiter\floor{\lfloor}{\rfloor}
%%%%%%%%%%%%%%%%%%%%%%%%%%%%%%%%%%%%%%%%%%%%%%%%

\usepackage[margin=1in, letterpaper]{geometry}
\newcommand{\problem}[1]{\bigskip\noindent\textbf{Problem #1}}
\newcommand{\ppart}[1]{\bigskip\textbf{(#1)}}
\newcommand{\lemma}{\bigskip\textbf{Lemma}}
\newcommand{\E}{\mathrm{E}}
\newcommand{\Var}{\mathrm{Var}}
\newcommand{\Cov}{\mathrm{Cov}}


Rousseau clearly identifies two distinct stages in the development of the human race, and human communities. He argues that there existed a different kind of human, in the "state of nature" (4), one who was not corrupted by the advancement of human civilisation, in particular that of modern leisure and industry. His main line of argument is that all the changes that were brought about by human progress - which he does not deem such - have impacted humanity in a negative way, and the concept of inequality among humans is explicitly tied to this and thus cannot be explained by natural law.

Rousseau states that humans, in the "state of nature", had no reason to interact with each other as much as they do now:

\begin{quote}
  It is in fact impossible to conceive why, in a state of nature, one man should stand more in need of the assistance of another, than a monkey or a wolf of the assistance of another of its kind: or, granting that he did, what motives could induce that other to assist him; or, even then, by what means they could agree about the conditions. (15)
\end{quote}

His main point is the idea that, in this natural state, humans are still unequal naturally, as not every human is exactly the same. However, he argues that human inequality as it exists today exists primarily due to the greatly increased exposure and influence of these differences, due in turn to the increase in interactions between humans, as a result of the advent of both leisure and industry. With regards to the former, he claims:

\begin{quote}
  Whoever sang or danced best, whoever was the handsomest, the strongest, the most dexterous, or the most eloquent, came to be of most consideration; and this was the first step towards inequality, and at the same time towards vice. (24)
\end{quote}

Rousseau nowhere denies that, had this dancing ritual not existed, some humans would be hypothetically better than others at it; he simply states that that difference would not be exposed. In doing so, he makes a great blunder, but one that can only be considered as such in a modern context. Rousseau's work needs to be read in its proper context of 1751, when many theories and abstractions that exist today had not been discovered or created. One such theory is that of evolution by natural selection, proposed by Charles Darwin in 1858, which states that the competition between living organisms is the main driving force behind their evolution (the improvement of heritable traits over time to better suit the organism's environment). The theory rationalises genetic diversity as a key component of this, allowing the species with the best genetic makeup to "win", and reproduce, passing on its heritable traits. Thus, evolution by natural selection accepts as natural that which Rousseau claims to be man-made: the inherent competition between different individuals of the same species, exacerbated by the inherent differences between them.

Perhaps a middle-ground standing would make more sense in this modern context, and this is a point that Rousseau unwittingly makes. Darwin's argument concerns the inequality that arises as a result of purely functional traits, for example, physical strength. Rousseau, later in his essay, elaborates that another component which became significant with the formation of society was the perception of humans by other humans: "these being the only qualities capable of commanding respect, it soon became necessary to possess or to affect them" (27). Moreover, he argues:

\begin{quote}
  It now became the interest of men to appear what they really were not. To be and to seem became two totally different things; and from this distinction sprang insolent pomp and cheating trickery, with all the numerous vices that go in their train. (27)
\end{quote}

This makes more sense in a modern context, since perception by others is not a functional trait in the way physical strength is, even though it is often heritable to a large extent, by means of many small characteristics. If we go back to Rousseau's example of "dancing" (24), we have this separate idea of "whoever was the handsomest, the strongest, the most dexterous … [coming] to be of most consideration". These traits are to a large extent heritable and evolutionarily advantageous under pure Darwinism, but, as Rousseau argues, they become doubly important in human society, by means of improving perception and standing among peers.

Karl Marx was potentially largely influenced by Rousseau's work, even though he almost never refers to the latter's work in his own. Rousseau was one of the first philosophers to suggest that private property was the source of inequality among humans, and that the latter was not naturally inherent. However, from a Marxist perspective, Rousseau is also somewhat wrong. Discourse explicitly claims that modern civilisation is incompatible with an egalitarian society, whereas Marx suggested that there is a way to reconcile the two, by fundamentally changing the system of government. On the other end of the spectrum, Discourse predated Adam Smith's Wealth of Nations by over 20 years, and it could be argued that in the context of the latter, Discourse makes some glaring errors. Rousseau argues that man is not naturally purely self-serving:

\begin{quote}
  Above all, let us not conclude, with Hobbes, that because man has no idea of goodness, he must be naturally wicked; that he is vicious because he does not know virtue; that he always refuses to do his fellow-creatures services which he does not think they have a right to demand; or that by virtue of the right he truly claims to everything he needs... (15)
\end{quote}

Rousseau backs this line of argument up by making a distinction between self-respect and egoism (second footnote on page 16). He argues that the former is naturally found in humans and the latter is not, and that the latter is the source of the problems that he describes. Smith took a very different approach: like Hobbes he contended that all humans are naturally self-serving, but he argued that this is not to their detriment but rather to their benefit. His principle of the "invisible hand" claims that, as each human acts only in his own interest, he unintentionally maximises the welfare of the whole of society: "he intends only his own gain, and he is in this, as in many other cases, led by an invisible hand to promote an end which was no part of his intention". This argument somewhat agrees with \textit{Discord}'s premise, but fundamentally disagrees with its conclusion.

\end{document}
