\documentclass[12pt]{article}
\usepackage{fancyhdr}     % Enhanced control over headers and footers
\usepackage[T1]{fontenc}  % Font encoding
\usepackage{mathptmx}     % Choose Times font
\usepackage{microtype}    % Improves line breaks
\usepackage{setspace}     % Makes the document look like horse manure
\usepackage{lipsum}       % For dummy text
\usepackage{etoolbox}

\AtBeginEnvironment{quote}{\singlespacing\small}

\newcommand{\name}{Kirill Chernyshov}

\pagestyle{fancy} % Default page style
\lhead{\name}
\chead{}
\rhead{\thepage}
\lfoot{}
\cfoot{}
\rfoot{}
\renewcommand{\headrulewidth}{1pt}
\renewcommand{\footrulewidth}{1pt}

\thispagestyle{empty} %First page style

\setlength\headheight{15pt} %Slight increase to header size

\begin{document}
\begin{center}
\begin{tabular}{c}
\textbf{\name} \\
\textbf{\today}
\end{tabular}
\end{center}
\doublespacing

\begin{document}
\title{Introduction to Nutritional Immunity}
\author{Charlie Cranston}
\maketitle

\section{Foreword}
Very few things are needed to be memorised for this topic. For example, metal metabolism is relatively simple, it mainly involves magnesium, zinc et cetera. The metal stays as an ion and is just transported to different places. The main topic of these lectures is how these metaals influence our immune system.

\section{Diseases}

\subsection{Haemochromatosis}
Heamochromatosis is an issue with iron metabolism. On the diagram, blue spots show abnormal iron deposits. Patients usually suffer various bacterial infections as a result.

[diagram]

\subsection{Menkes disease}
Menkes comes from a problem with copper metabolism, and sufferers usually don't make it past infancy. One of the symptoms is coil shaped hair; this is because an enzyme needed to make collagen requires copper. As with haemochromatosis, it is usually accompanied by many bacterial infections.

\section{Nutritional immunity}
Nutritional immunity is to do with the interactions between bacteria and the human immune system. When infecting a host, the bacteria rely on the host for nutrients that they need, and as a result there is a conflict between the host and the bacteria for the control of nutrients.

\subsection{Resistance}
Bacterial infections are currently a huge problem, as many strains are resistant, and some strains are resistant to everything available. It is estimated that by 2050, AMR-resistant bacterial infections will be the largest cause of death worldwide.

As a result, a multi-strategy approach is often used for maximum effect.

\subsection{Acute-phase response}
Manganese, iron, cobalt, copper, zinc, molybdenum and selenium are some metals that are required by all cells in the body. In response to an infection, the host usually stores these metals away from the bacteria to make them less easily available, with the exception of copper. This restriction of metals also occurs at tissue level. Thanks to a special version of mass spectrometry, we can observe the levels of metals found in different areas of the liver, and see this response in action. The spot pointed to by the arrow is an abscess formed as a result of a \textit{Staph. Aureus} infection. For example, the zinc ions can be seen to have been moved away from the infected area.

[diagram]

The balance of metal levels is important, since the host needs enough to survive, but it must also make sure to not make it easily accessible to bacteria in case of infection. Host cells will try to allow themselves to take up metals so as to not let bacteria take them up.

\subsection{Link with redox biochemistry}
Phagocytes use antimicrobial molecules to kill bacteria by means of redox reactions. RNS and ROS are analogous, with focus on nitrogen and oxygen respectively.

\end{document}
